\documentclass[12pt]{book}

\usepackage[utf8]{inputenc} % Required for inputting international characters
\usepackage[T1]{fontenc} % Output font encoding for international characters

\usepackage{comment}
\usepackage{mathpazo} % Palatino font
\usepackage{amssymb}
\usepackage{amsmath}
\usepackage{array}
\usepackage{amsthm}
\usepackage{graphicx}
\usepackage{multirow}
\usepackage{relsize}
\usepackage{hyperref}
\usepackage{xcolor}
\hypersetup{
    colorlinks,
    linkcolor={red!50!black},
    citecolor={blue!50!black},
    urlcolor={blue!80!black}
}
\usepackage[makeroom]{cancel}

\usepackage[margin=1in]{geometry}
\newcommand{\overbar}[1]{\mkern 1.5mu\overline{\mkern-1.5mu#1\mkern-1.5mu}\mkern 1.5mu}

\newcommand*\conj[1]{\overbar{#1}}
% STATS Shortcut
\newcommand{\BP}{\Bbb P}
\let\Bbb\mathbb
\def\sep{\phantom{}}
\theoremstyle{definition}
\newtheorem{definition}{Définition}[section]
\newtheorem*{example}{Exemple}
\newtheorem{theorem}{Theorême}[section]
\newtheorem{corollary}{Corollaire}[theorem]
\newtheorem{lemma}[theorem]{Lemme}

\title{Analyse des ℝ}

\begin{document}
\chapter{Définition préalable}
Ici se retrouve un ensemble de définition préalable à la compréhension du livre:
\begin{definition}[Asymptote]
    \label{def:asymptote}
    Droite tangente à une courbe; La courbe semble s'approcher alors que l'on approche l'infini mais ne finira
    jamais par toucher la droite.
\end{definition}

\chapter{La Continuité}
\section{Définition}
Initialement, lorsque l'on approche le concept de continuité de fonction dans certains cours de mathématiques au secondaire ou au 
Cegep, on présente la continuité comme une propriété d'une fonction dans le but de \textit{la décrire}. Cette propriété est ensuite
utilisée comme critère à bien d'autres résultats subséquents.

On classifie généralement les fonctions en fonctions \textbf{continues} et en fonctions \textbf{discontinues}. Dans l'absence de précision
il faut assumer que le terme "fonction continue" réfère à une fonction continue sur son domaine en entier.

La description intuitive d'une fonction continue est généralement celle-ci: Faites le graphe d'une fonction, si vous arrives à tracer
la fonction sans jamais levée le crayon, alors celle-ci est continue. C'est cette description qui nous permet d'intuitivement reconnaître si la fonction est continue d'un simple coup d'oeil;
S'il n'y a pas de saut, c'est continue! Ainsi, on remarque que la fonction $f(x) = 2x$ est continue.

Parallèlement, on remarque que la fonction $f(x) = \frac{1}{x}$ ne semble pas être continue puisqu'il y a une \hyperref[def:asymptote]{asymptote} en $x = 0$.
Toutefois, c'est ici que la description intuitive de la fonction continue semble confondre. En effet, la fonction $f(x) = \frac{1}{x}$ \textit{n'est pas défini} en $x = 0$.
Si on ne regarde que les points du domaine (qui exclut 0), la fonction est continue!

En plus de l'ambiguité de notre définition intuitive, nous sommes limités au $\Bbb{R}$. Qu'en est-il des fonctions continues dans $\Bbb{C}$ ou dans $\Bbb{R}^n$?

Voilà pourquoi il est nécessaire de donner une définition plus rigoureuse à la continuité avec laquelle il sera plus pratique de travailler. Commençons
par définir la continuité en un point:
\begin{definition}[Continuité en un point]
    \label{def:continue-1-point}
    Soit $x_0 \in D_f$, $f(x)$ est \textbf{continue} au point $x_0$, si $(\forall \epsilon > 0)(\exists \delta > 0)\text{ tq }\forall x \in (D_f \cap V(x_0, \delta)) \implies f(x) \in V(f(x_0), \epsilon)$
\end{definition}
Prenons une fonction linéaire que l'on sait être continue:
\begin{example}[Pour $x=1$ et $f(x) = x$]
    Soit $f(x) = x, x_0 = 1$, $1 \in D_f$, $f(1) = 1$. \sep Posons $\delta = \epsilon$, \sep Donc $\forall x \in (D_f \cap V(1, \epsilon)) \implies f(x) \in V(f(1), \epsilon)$
\end{example}
\begin{comment}
    On part avec nos hypothèses, on cherche à voir si au point x=1 ($x_0$), la fonction est continue. \sep
    Toujours dans l'idée que lorsque l'on cherche une proposition y de vraie pour tout x, l'idéal est de poser y en fonction de x \sep
    $f(x) = x$ et $f(1) = 1$, donc on ne fait que remplacer $\delta$ par $\epsilon$ et $f(x)$ par $x$ pour constater que c'est vrai.
\end{comment}
Maintenant, tentons de définir la discontinuité:
\begin{definition}[Discontinuité en un point]
    \label{def:discontinue-1-point}
    Une fonction est \textbf{discontinue} en un point, si elle n'est pas
\end{definition}

Cette propriété, nous amène à catégoriser les fonctions en fonctions \textbf{continues} et en fonctions \textbf{discontinues}. 
Par la suite, on utilise cette propriété comme critère pour 
Boo, this is some more text \textbf{Bold text} Indeed. \\
\\
Wow.
\subsection{Sub section 1}
This is some text. And while this not  OH
the end of the world\dots ok?


This SHOULD be another paragraph, right? Oh and this is google: \href{http://google.com}{Google}
\begin{theorem}
    \label{thm:continuity}
    Let $f$ be a function whose derivative exists    in every point,
    then $f$ is a continuous function.
\end{theorem}
\begin{definition}[Boo]
    Ceci est une \textit{définition}.
\end{definition}
\begin{example}
    Ceci est un example.
\end{example}
\begin{proof}
    To prove it by contradiction try and assume that \textbf{the statement} is false, \sep
    proceed from there and at some point you will arrive to a contradiction like in \autoref{thm:continuity}.
\end{proof}
\begin{comment}
    This is a comment, \sep
    a multi-line comment,
\end{comment}
En effet, tel qu'on a vu \hyperref[thm:continuity]{précédemment.}
\begin{proof}
    This is another proof
\end{proof}
\begin{comment}
    This is a comment.
\end{comment}
\begin{lemma}
    Ceci est un lemme.
\end{lemma}
\begin{theorem}[Theorême de Pythagore]
    Let $f \forall x$ be a function whose derivative exists in every point,
    then $f$ is a continuous function. 
\end{theorem}
\begin{corollary}
    Ceci est un corollaire.
\end{corollary}
\section{Section 2}
OOOOH

\chapter{La vie après la mort}
Ceci est un paragraphe d'introduction.
\chapter{La mort après la vie}
\section{Section 1}
Oh Damn. \autoref{thm:continuity}

\end{document}