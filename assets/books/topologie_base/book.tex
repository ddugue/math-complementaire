\documentclass[12pt]{book}

\usepackage[utf8]{inputenc} % Required for inputting international characters
\usepackage[T1]{fontenc} % Output font encoding for international characters

\usepackage{comment}
\usepackage{mathpazo} % Palatino font
\usepackage{amssymb}
\usepackage{amsmath}
\usepackage{array}
\usepackage{amsthm}
\usepackage{graphicx}
\usepackage{multirow}
\usepackage{relsize}
\usepackage{hyperref}
\usepackage{xcolor}
\hypersetup{
    colorlinks,
    linkcolor={red!50!black},
    citecolor={blue!50!black},
    urlcolor={blue!80!black}
}
\usepackage[makeroom]{cancel}

\usepackage[margin=0.5in]{geometry}
\newcommand{\overbar}[1]{\mkern 1.5mu\overline{\mkern-1.5mu#1\mkern-1.5mu}\mkern 1.5mu}

\newcommand*\conj[1]{\overbar{#1}}
% STATS Shortcut
\newcommand{\BP}{\Bbb P}
\newcommand{\BR}{\Bbb R}
\let\Bbb\mathbb
\def\sep{\phantom{}}
\newcommand\todo[1]{\phantom{#1}}
\theoremstyle{definition}
\newtheorem{definition}{Définition}[section]
\newtheorem*{example}{Exemple}
\newtheorem{theorem}{Theorême}[section]
\newtheorem{corollary}{Corollaire}[theorem]
\newtheorem{lemma}[theorem]{Lemme}  

\title{Topologie}

\begin{document}
\chapter{Introduction}
Qu'est-ce que la topologie? C'est l'Analyse d'espaces abstraits. Toutefois, l'objectif de la topologie est surtout d'analyser la continuité de ces 
espaces entre eux et sur eux. En ayant une continuité on peut alors modifier l'espace, parfois infini, d'une manière satisfaisante.

L'idée de ce "livre" est d'initier aux différentes définitions topologiques généralisés. Il sera ainsi possible de réutiliser ces définitions 
et ces explications abstraites lors de l'analyse des différents types d'espaces topologiques.
\chapter{Espaces topologiques}
\section{Topologie}
\begin{definition}[Topologie]
    \label{def:topologie} 
    Soit $\mathcal{O}$, une \hyperref[def:famille]{famille} est une \textbf{topologie} sur $X$ si et seulement si:\\
    $(x \in \mathcal{O} \implies x \subset X)$ (chaque élément de la famille est un sous-ensemble de $X$)\\
    $(\emptyset \in \mathcal{O} \land X \in \mathcal{O})$ \\
    $(x \subset \mathcal{O} \land y \subset \mathcal{O} \implies x \cap y \in \mathcal{O})$ (si $x,y$ sont finis) \\
    $(x \subset \mathcal{O} \land y \subset \mathcal{O} \implies x \cup y \in \mathcal{O})$ \\
\end{definition}
En d'autres mots, c'est la famille de tous les ensembles qu'on peut former de $X$ lesquels peuvent se réunir ou s'intersecter et 
générer un ensemble qui fait partie de la famille.

Bref, il existe une multitude de façon de générer une topologie pour un ensemble. Si on prend par exemple la topologie 
pour $X = \{1,2,3,4\} \text{ tq } \mathcal{O} = \{\emptyset, X\}$, on peut voir que celle-ci correspond à la définition de 
notre topologie ci-dessus. Toutefois cette topologie (appellée la topologie triviale) est généralement peu intéressante.

Particulièrement, on peut définir, ce qu'en entend par espace topologique:
\begin{definition}[Espace topologique]
    \label{def:espace_topologique}
    Un \textbf{espace topologique} est un ensemble $X$ muni d'une \hyperref[def:topologie]{topologie} (sur $X$)
\end{definition}

\section{Ouverture}
Un concept clé dans une topologie est le concept d'ensembles ouverts et fermés. La définition d'un ensemble ouvert dans 
une topologie est la suivante:
\begin{definition}[Ensemble ouvert]
    \label{def:topologie_ouvert}
    Soit $\mathcal{O}$, une topologie sur $X$. Un \textbf{ensemble ouvert} est un ensemble $O$, tel que $O \in \mathcal{O}$.
\end{definition}
Bref, la définition d'un ensemble ouvert dans une topologie est plus ou moins tautologique. Il est important donc 
de comprendre que la définition de ce qu'est un ensemble ouvert est directement liée à comment on défini notre topologie.

\section{Fermeture}
Parallèlement, on peut maintenant définire la notion d'un ensemble fermé.
\begin{definition}[Ensemble fermé]
    \label{def:topologie_ferme}
    Soit $\mathcal{O}$ une topologie sur $X$. Un \textbf{ensemble fermé} est un ensemble $F$, tel que $F \subseteq X$ et
    l'exclusion de $F$ sur $X$, $X / F \in \mathcal{O}$ est un \hyperref[def:topologie_ouvert]{ensemble ouvert}.
\end{definition}

En d'autres mots, un ensemble est dit fermé si son complément $F^c$ est un \hyperref[def:topologie_ouvert]{ensemble ouvert}.
Notons toutefois qu'il est possible pour un ensemble d'être à la fois ouvert et fermé (un ensemble fouvert), tout comme il est 
possible pour un ensemble d'être ni fermé, ni ouvert.

\chapter{Continuité}
L'idée intéressante et cruciale de la topologie, c'est-à-dire la continuité, a d'abord été développée afin de généraliser 
la continuité dans les espaces métriques. On est donc partie de la définition de continuité dans un espace métrique pour 
appliquer une version encore plus générale de la continuité qui englobait cette définition de continuité dans un espace métrique.

\section{Définition générale}
Ainsi, on est arrivé à développer cette nouvelle définition uniquement basé sur la continuité:
\begin{definition}
    Soit $X$ et $Y$, deux espaces munis d'une \hyperref[def:topologie]{topologie} $\tau_X$ et $\tau_Y$,et $f: X \to Y$.
    Alors $f$ est \textbf{continue} $\iff$ $t_Y \in \tau_Y \implies f^{-1}(t_y) \in \tau_X$
\end{definition}
\chapter{Compacité}
Qu'entend-t-on par la "Compacité" d'un espace ou d'un sous-espace?
\chapter{Connexité}

\appendix
\chapter{Notions préalables}
Voici une série de définition importantes à considérer lors de la lecture de cet ouvrage. Celles-ci servent
de références pour les exemples et les preuves qui seront amenés dans les chapitres suivants. Pour une explication
plus complète et rigoureuse, il sera nécessaire de se réferrer aux livre précédent: Analyse des Réels.

\begin{definition}[Voisinage dans les Réels]
    \label{def:voisinage_reels}
    Ensemble de point satisfaisant l'expression
    suivante: $$V(a, \delta) = \{ x \in \Bbb R : |x - a| < \delta \}$$
\end{definition}

\begin{definition}[Voisinage troué dans les Réels]
    \label{def:voisinage_troue_reels}
    Ensemble de point satisfaisant l'expression
    suivante: $$V'(a, \delta) = \{ x \in \Bbb R : |x - a| < \delta \land x \neq a \}$$
\end{definition}

\begin{definition}[Points intérieurs de E]
    \label{def:point_int}
    Ensemble de points satisfaisant l'expression
    suivante: $$int(E) = \{ x \in E : \exists \delta > 0, V(x, \delta) \subseteq E \} $$
\end{definition}

\begin{definition}[Ensemble ouvert]
    \label{def:ensemble_ouvert}
    Un ensemble est dit ouvert si tous ces éléments sont des \hyperref[def:point_int]{points intérieurs.}
\end{definition}

\begin{definition}[Groupe abélien]
    \label{def:groupe_abelien}
    Un \textbf{groupe abélien} ou \textbf{groupe non-commutatif} est un groupe
    (une structure algébrique associative avec un élément neutre et un élément inverse $(\forall a \in G, \exists a^{-1})$), 
    dont l'opération binaire est commutative.
\end{definition}

\begin{definition}[Famille d'ensemble]
    \label{def:famille}
    Une \textbf{famille} ou \textbf{collection} est simplement un ensemble d'ensemble.
\end{definition}
\todo{Add reference}

\chapter{Cheatsheet}
\section{Formules}
\subsection{Inégalité de Cauchy-Schwarz}
$$(\sum_{i=1}^{n} a_ib_i)^2 \leq \sum_{i=1}^{n} a_i^2 \cdot \sum_{i=1}^{n} b_i ^ 2$$ 
\subsection{Convergence série géométrique}
$$|r| < 1 \implies \sum_{n=0}^{\infty} r^n \to \frac{1}{1 - r} $$
\subsection{Inégalité du triangle}
$$| ||x|| - ||y|| | \leq ||x \pm y|| \leq ||x|| + ||y|| $$
\section{Types d'espace}
\begin{itemize}
    \item Espace vectoriel: Ensemble d'éléments appellés vecteurs, muni de l'addition et de la multiplication par un scalaire.
    \item Espace préhilbertien: Espace vectoriel muni d'un produit scalaire.
    \item Espace normé: Espace vectoriel muni d'une norme.
    \item Espace métrique: Ensemble \textbf{quelconque} munie d'une métrique.
    \item Espace métrique complet:
    \item Espace topologique:
\end{itemize}
\section{Fonctions spéciales}
\subsection{Produit scalaire}
\begin{itemize}
    \item Un produit scalaire est une fonction de $V \times V \to \Bbb R$, noté $\langle x, y \rangle$ avec 5 propriétés qui la définisse
    \item $\langle x, x \rangle \geq 0$
    \item $\langle x, x \rangle = 0 \iff x = 0$
    \item $\langle x, y \rangle = \langle y, x \rangle$
    \item $\langle x + y, z\rangle = \langle x, z \rangle + \langle y, z \rangle$
    \item $\alpha\langle x, y \rangle = \langle \alpha x, y \rangle \forall \alpha \in \Bbb R$
\end{itemize}
\subsection{Norme}
\begin{itemize}
    \item Une norme sur $V$ est une fonction $V$ vers $\Bbb R$, noté $x \mapsto ||x||$ avec 4 propriétés qui la définisse:
    \item $||x|| \geq 0$
    \item $||x|| = 0 \iff x = 0$
    \item $||\alpha x|| = |\alpha| \cdot ||x||$
    \item $||x + y || \leq ||x|| + ||y||$
    \item Norme euclidienne: $\sqrt{x_1^2 + x_2^2 + \dots + x_n^2}$
    \item Le produit scalaire est une norme.
\end{itemize}
\subsection{Métrique}
\begin{itemize}
    \item Une métrique est une fonction $X \times X \to 0, \infty$ avec les 4 propriétés suivantes:
    \item $d(x,y) \geq 0$
    \item $d(x,y) = d(y, x)$
    \item $d(x,y) = 0 \iff x = y$
    \item $d(x,y) \leq d(x, z) + d(z, y)$
    \item Métrique discrète: $d(x,y) = 0$ si $x = y$, 1 si $x \neq 1$
    \item La norme est une métrique
\end{itemize}
\section{Définition topologiques}
\begin{itemize}
    \item \textbf{Boule ouverte} : $B(x, r) = \{y \in R^n : || x - y || < r\}, r > 0$
    \item \textbf{Boule fermée} : $B(x, r) = \{y \in R^n : || x - y || \leq r\}, r > 0$
    \item \textbf{Point intérieur}: $x \in \Bbb R^n \text{ est un point intérieur de } E \iff \exists \delta > 0 : B(x, \delta) \subset E$
    \item \textbf{Ensemble ouvert}: $E\text{ est ouvert} \iff E = int(E)$
    \item \textbf{Convergence suite}: $\forall \epsilon > 0, \exists M \in \Bbb N \text{ t q} m \geq M \implies d(x_m, x) < \epsilon$
    \item \textbf{Ensemble borné}: $\exists x_0 \in X \land \exists r > 0, \forall x \in A, d(x_0, x) < r$
\end{itemize}

\end{document}