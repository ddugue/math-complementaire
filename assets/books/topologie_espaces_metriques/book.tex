\documentclass[12pt]{book}

\usepackage[framemethod=TikZ]{mdframed}
\usepackage[utf8]{inputenc} % Required for inputting international characters
\usepackage[T1]{fontenc} % Output font encoding for international characters

\usepackage{comment}
\usepackage{mathpazo} % Palatino font
\usepackage{amssymb}
\usepackage{amsmath}
\usepackage{array}
\usepackage{amsthm}
\usepackage{graphicx}
\usepackage{multirow}
\usepackage{relsize}
\usepackage{hyperref}
\usepackage{xcolor}
\hypersetup{
    colorlinks,
    linkcolor={red!50!black},
    citecolor={blue!50!black},
    urlcolor={blue!80!black}
}
\usepackage[makeroom]{cancel}

\usepackage[margin=0.5in]{geometry}
\newcommand{\overbar}[1]{\mkern 1.5mu\overline{\mkern-1.5mu#1\mkern-1.5mu}\mkern 1.5mu}

\newcommand*\conj[1]{\overbar{#1}}
% STATS Shortcut
\newcommand{\BP}{\Bbb P}
\newcommand{\BR}{\Bbb R}
\let\Bbb\mathbb
\def\sep{\phantom{}}
\newcommand\todo[1]{\phantom{#1}}
\theoremstyle{definition}
\newtheorem{definition}{Définition}[section]
\newtheorem*{example}{Exemple}
\newtheorem{theorem}{Theorême}[section]
\newtheorem{corollary}{Corollaire}[theorem]
\newtheorem{lemma}[theorem]{Lemme}  

\linespread{1.5} %regulate line spacing

\title{Topologie}

\begin{document}
\chapter{Introduction}
Qu'est-ce que la topologie?

\chapter{Complétude}
\begin{definition}[Espace métrique complet]
    \label{def:completude}
    Un espace métrique est dit \textbf{complet} si toute suite de Cauchy dans l'espace métrique converge vers un élément de l'espace métrique.
\end{definition}

\chapter{Compacité}
\begin{definition}[Compacité]
    \label{def:compacite}
    Soit $S$, un sous-ensemble d'un espace métrique $(X,d)$. On dit que $S$ est \textbf{compact} si de tout recouvrement de $S$ par des ouverts de $X$,
    on peut extraire un sous-recouvrement \underline{fini}.
\end{definition}

\begin{theorem}
    Soit $(X,d)$ un espace métrique. Si $S \subseteq X$ est \hyperref[def:compacite]{compact}, alors $S$ est borné et fermé.
\end{theorem}

\begin{theorem}
    \label{thm:compact_implique_sous_suite}
    Soit $(X,d)$, un espace métrique compact. Alors, toute suite ${x_n}$ de l'espace métrique admet une sous-suite convergente 
\end{theorem}

\begin{theorem}
    Tout espace métrique compact est un espace métrique \hyperref[def:completude]{complet}.
\end{theorem}

\begin{proof}
    Soit $X$ un espace métrique \hyperref[def:completude]{compact} et ${x_n}$ une suite de Cauchy d'éléments de $X$.\\
    Ainsi, $\forall \epsilon > 0$, $\exists N_1\text{ tq } m,n \geq N_1 \implies d(x_n, x_m) < \frac{\epsilon}{2}$.\\
    Par le \autoref{thm:compact_implique_sous_suite},  il existe aussi une sous-suite convergente ${x_{n_k}} \mapsto x \in X$.\\
    Ainsi $\forall \epsilon > 0$, $\exists N_2\text{ tq } k \geq N_2 \implies d(x_{n_k}, x) < \frac{\epsilon}{2}$. \\
    En posant $N_3 = \max(N_1, N_2)$, alors:\\
    $\forall \epsilon > 0$, $\exists N_3\text{ tq } n,k \geq N_3$ \\
    $\implies d(x_{n_k}, x) < \frac{\epsilon}{2} \land d(x_n,x_{n_k}) < \frac{\epsilon}{2}$\\
    $\implies d(x_{n_k}, x) + d(x_n, x_{n_k}) < \frac{\epsilon}{2} + \frac{\epsilon}{2}$\\
    $\implies d(x_n,x) \geq d(x_{n_k}, x) + d(x_n, x_{n_k}) < \epsilon$\\
    Et donc que toute suite de cauchy converge vers un $x \in X$.
\end{proof}

\begin{theorem}
    Si $f$, est une fonction continue surjective de l'espace métrique compact $(X,d)$ sur l'espace métrique $(Y, d')$, alors Y est compact.
\end{theorem}
\begin{proof}
    Posons $Y \subseteq \bigcup{V_i}$ où $V_i$ sont des ensembles ouverts de $Y$. \\
    Puisque $f$ est une fonction surjective, $\forall x \in X, f(x) \in Y$.\\
    Ainsi $X \subseteq f^{-1}(Y) \subseteq f^{-1}(\bigcup{V_i})$ ou $X \subseteq \bigcup f^{-1}(V_i)$\\
    Comme $f$ est continue $f^{-1}(V_i)$ est ouvert.\\
    Alors $X$ est recouvert par le recouvrement ouvert $\bigcup f^{-1}(V_i)$\\
    Puisque $X$ est compact, on peut extraire un sous-recouvrement fini tel que $X = \bigcup^n_{j=1} f^{-1}(V_{i_j})$.\\
    Comme $Y = f(X) \implies Y = f(\bigcup^n_{j=1} f^{-1}(V_{i_j}))$ \\
    $ \implies Y = \bigcup^n_{j=1} f(f^{-1}(V_{i_j})) $\\
    $ \implies Y = \bigcup^n_{j=1} V_{i_j}$, car $f$ est surjective.\\
    Donc, il existe un recouvrement fini d'ensemble ouvert pour $Y$.
\end{proof}

Montrons de deux façons différentes que si en espace métrique est borné et fermé, alors il n'est pas nécessairement compact:

\begin{example}
    Prenons l'espace métrique discret borné et fermé tel $x \in [0, 1]$. Prenons le recouvrement $[0,1] \subseteq \bigcup_i\in V_i$ où $V_i = \{x_i\}$ donc 
    un singleton $\{x_i\}$ tel que $0 \leq x_i \leq 1$. Il est toutefois impossible de former un recouvrement fini de ce même ensemble, car si $\bigcup_i\in V_i / {x_j}$
    ça ne recouvrira pas l'ensemble.
\end{example}

\begin{example}
    Prenons l'es
\end{example}
\chapter{Connexité}
\begin{definition}[Connexité]
    \label{def:connexite}
    Soit $(X,d)$, un espace métrique. On dit que $(X,d)$ est connexe si il ne peut s'exprimer comme l'union de deux ouverts disjoints non-vides.
\end{definition}

\begin{theorem}
    Si $f$, est une fonction continue surjective de l'espace métrique connexe $(X,d)$ sur l'espace métrique $(Y, d')$, alors Y est connexe.
\end{theorem}

\begin{proof}
    Supposons que $X$ soit connexe et que $Y$ s'exprime comme l'union de deux ensembles ouverts disjoints $(U_1 \cup U_2)$.\\
    $X = f^{-1}(Y) = f^{-1}(U_1 \cup U_2) = f^{-1}(U_1) \cup f^{-1}(U_2)$\\
    Puisque $f$ est continue, l'image de $f$ est ouvert. \\
    Ainsi $f^{-1}(U_1)$ et $f^{-1}(U_2)$ sont ouverts.\\
    De plus, puisque $U_1 \cap U_2 = \emptyset \implies f^{-1}(U_1) \cap f^{-1}(U_2) = \emptyset$\\
    Ensuite $U_1 \neq \emptyset \implies f^{-1}(U_1) \neq \emptyset$ puisque $f$ est surjectif\\
    Donc cela implique que $X$ ne serait pas connexe, une contradiction.
\end{proof}

\begin{theorem}
    Tout espace métrique connexe par arc est connexe.
\end{theorem}

\begin{proof}
    Supposons que $X$ soit connexe par arcs, mais qu'il ne soit pas connexe.\\\sep
    Ainsi $X = U_1 \cup U_2$, $U_1, U_2$ ouverts et $U_1 \neq \emptyset \neq U_2$ et $U_1 \cap U_2 = \emptyset$. \\\sep
    Soit $x_1 \in U_1$ et $x_2 \in U_2$. \\\sep
    Puisque $X$ est connexe par arc il existe un arc tq :$[0,1] \to K \subseteq X$ et $a(0) = x_1$ et $a(1) = x_2$. \\\sep 
    Puisque $[0,1]$ est connexe et $f$ continue et surjective sur $K$, alors $K$ est connexe. \\\sep 
    Mais $K = K \cap X = K \cap (U_1 \cup U_2) = (K \cap U_1) \cup (K \cap U_2)$. \\\sep 
    Puisque l'intersection d'un ouvert avec un autre ensemble est ouvert, $K$ s'exprime comme l'union de deux ensembles ouverts. \\\sep 
    De plus $K \cap U_2 \neq \emptyset$, car $x_2 \in U_2$ \\\sep 
    et $(K \cap U_1) \cap (K \cap U_2) = \emptyset$, car $U_1 \cap U_2 = \emptyset$ \\\sep 
    Ainsi $K$ ne serait pas connexe, en contradiction.
\end{proof}
\appendix
\chapter{Notions préalables}
Voici une série de définition importantes à considérer lors de la lecture de cet ouvrage. Celles-ci servent
de références pour les exemples et les preuves qui seront amenés dans les chapitres suivants. Pour une explication
plus complète et rigoureuse, il sera nécessaire de se réferrer aux livre précédent: Analyse des Réels.

\begin{definition}[Voisinage dans les Réels]
    \label{def:voisinage_reels}
    Ensemble de point satisfaisant l'expression
    suivante: $$V(a, \delta) = \{ x \in \Bbb R : |x - a| < \delta \}$$
\end{definition}

\begin{definition}[Voisinage troué dans les Réels]
    \label{def:voisinage_troue_reels}
    Ensemble de point satisfaisant l'expression
    suivante: $$V'(a, \delta) = \{ x \in \Bbb R : |x - a| < \delta \land x \neq a \}$$
\end{definition}

\begin{definition}[Points intérieurs de E]
    \label{def:point_int}
    Ensemble de points satisfaisant l'expression
    suivante: $$int(E) = \{ x \in E : \exists \delta > 0, V(x, \delta) \subseteq E \} $$
\end{definition}

\begin{definition}[Ensemble ouvert]
    \label{def:ensemble_ouvert}
    Un ensemble est dit ouvert si tous ces éléments sont des \hyperref[def:point_int]{points intérieurs.}
\end{definition}

\begin{definition}[Groupe abélien]
    \label{def:groupe_abelien}
    Un \textbf{groupe abélien} ou \textbf{groupe non-commutatif} est un groupe
    (une structure algébrique associative avec un élément neutre et un élément inverse $(\forall a \in G, \exists a^{-1})$), 
    dont l'opération binaire est commutative.
\end{definition}

\begin{definition}[Famille d'ensemble]
    \label{def:famille}
    Une \textbf{famille} ou \textbf{collection} est simplement un ensemble d'ensemble.
\end{definition}
\todo{Add reference}

\chapter{Cheatsheet}
\section{Formules}
\subsection{Inégalité de Cauchy-Schwarz}
$$(\sum_{i=1}^{n} a_ib_i)^2 \leq \sum_{i=1}^{n} a_i^2 \cdot \sum_{i=1}^{n} b_i ^ 2$$ 
\subsection{Convergence série géométrique}
$$|r| < 1 \implies \sum_{n=0}^{\infty} r^n \to \frac{1}{1 - r} $$
\subsection{Inégalité du triangle}
$$| ||x|| - ||y|| | \leq ||x \pm y|| \leq ||x|| + ||y|| $$
\section{Types d'espace}
\begin{itemize}
    \item Espace vectoriel: Ensemble d'éléments appellés vecteurs, muni de l'addition et de la multiplication par un scalaire.
    \item Espace préhilbertien: Espace vectoriel muni d'un produit scalaire.
    \item Espace normé: Espace vectoriel muni d'une norme.
    \item Espace métrique: Ensemble \textbf{quelconque} munie d'une métrique.
    \item Espace métrique complet:
    \item Espace topologique:
\end{itemize}
\section{Fonctions spéciales}
\subsection{Produit scalaire}
\begin{itemize}
    \item Un produit scalaire est une fonction de $V \times V \to \Bbb R$, noté $\langle x, y \rangle$ avec 5 propriétés qui la définisse
    \item $\langle x, x \rangle \geq 0$
    \item $\langle x, x \rangle = 0 \iff x = 0$
    \item $\langle x, y \rangle = \langle y, x \rangle$
    \item $\langle x + y, z\rangle = \langle x, z \rangle + \langle y, z \rangle$
    \item $\alpha\langle x, y \rangle = \langle \alpha x, y \rangle \forall \alpha \in \Bbb R$
\end{itemize}
\subsection{Norme}
\begin{itemize}
    \item Une norme sur $V$ est une fonction $V$ vers $\Bbb R$, noté $x \mapsto ||x||$ avec 4 propriétés qui la définisse:
    \item $||x|| \geq 0$
    \item $||x|| = 0 \iff x = 0$
    \item $||\alpha x|| = |\alpha| \cdot ||x||$
    \item $||x + y || \leq ||x|| + ||y||$
    \item Norme euclidienne: $\sqrt{x_1^2 + x_2^2 + \dots + x_n^2}$
    \item Le produit scalaire est une norme.
\end{itemize}
\subsection{Métrique}
\begin{itemize}
    \item Une métrique est une fonction $X \times X \to 0, \infty$ avec les 4 propriétés suivantes:
    \item $d(x,y) \geq 0$
    \item $d(x,y) = d(y, x)$
    \item $d(x,y) = 0 \iff x = y$
    \item $d(x,y) \leq d(x, z) + d(z, y)$
    \item Métrique discrète: $d(x,y) = 0$ si $x = y$, 1 si $x \neq 1$
    \item La norme est une métrique
\end{itemize}
\section{Définition topologiques}
\begin{itemize}
    \item \textbf{Boule ouverte} : $B(x, r) = \{y \in R^n : || x - y || < r\}, r > 0$
    \item \textbf{Boule fermée} : $B(x, r) = \{y \in R^n : || x - y || \leq r\}, r > 0$
    \item \textbf{Point intérieur}: $x \in \Bbb R^n \text{ est un point intérieur de } E \iff \exists \delta > 0 : B(x, \delta) \subset E$
    \item \textbf{Ensemble ouvert}: $E\text{ est ouvert} \iff E = int(E)$
    \item \textbf{Convergence suite}: $\forall \epsilon > 0, \exists M \in \Bbb N \text{ t q} m \geq M \implies d(x_m, x) < \epsilon$
    \item \textbf{Ensemble borné}: $\exists x_0 \in X \land \exists r > 0, \forall x \in A, d(x_0, x) < r$
\end{itemize}

\end{document}