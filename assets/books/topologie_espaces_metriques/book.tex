\documentclass[12pt]{book}

\usepackage[utf8]{inputenc} % Required for inputting international characters
\usepackage[T1]{fontenc} % Output font encoding for international characters

\usepackage{comment}
\usepackage{mathpazo} % Palatino font
\usepackage{amssymb}
\usepackage{amsmath}
\usepackage{array}
\usepackage{amsthm}
\usepackage{graphicx}
\usepackage{multirow}
\usepackage{relsize}
\usepackage{hyperref}
\usepackage{xcolor}
\hypersetup{
    colorlinks,
    linkcolor={red!50!black},
    citecolor={blue!50!black},
    urlcolor={blue!80!black}
}
\usepackage[makeroom]{cancel}

\usepackage[margin=0.5in]{geometry}
\newcommand{\overbar}[1]{\mkern 1.5mu\overline{\mkern-1.5mu#1\mkern-1.5mu}\mkern 1.5mu}

\newcommand*\conj[1]{\overbar{#1}}
% STATS Shortcut
\newcommand{\BP}{\Bbb P}
\newcommand{\BR}{\Bbb R}
\let\Bbb\mathbb
\def\sep{\phantom{}}
\newcommand\todo[1]{\phantom{#1}}
\theoremstyle{definition}
\newtheorem{definition}{Définition}[section]
\newtheorem*{example}{Exemple}
\newtheorem{theorem}{Theorême}[section]
\newtheorem{corollary}{Corollaire}[theorem]
\newtheorem{lemma}[theorem]{Lemme}  

\title{Topologie}

\begin{document}
\chapter{Introduction}
Qu'est-ce que la topologie?


\chapter{Espaces vectoriels}
\todo{Complete}
\begin{definition}
    \label{def:espace_vectoriel} 
    Soit $K$ un corps commutatif, Un \textbf{espace vectoriel} est un ensemble de vecteurs $V$ muni de deux opérations binaires
\end{definition}

\section{Addition de vecteurs}
\begin{itemize}
    \item ao
\end{itemize}

\section{Multiplication par un scalaire}
\begin{itemize}
    \item ao
\end{itemize}

\section{Produit scalaire}
\begin{definition}
    \label{def:produit_scalaire}
\end{definition}
\subsection{Produit scalaire dans R}
\begin{definition}
    \label{def:produit_scalaire_r}
\end{definition}
\subsection{Produit scalaire dans Rn}
\begin{definition}
    \label{def:produit_scalaire_rn}
\end{definition}


\chapter{Inégalité de Cauchy-Schwarz}
\begin{lemma}[Inégalité de Cauchy-Schwarz]
    Soit $a_1,a_2,\dots,a_n$ et $b_1,b_2,\dots,b_n$ des nombre réels arbitraires, alors:
    $$ (\sum_{i=1}^{n}a_ib_i)^2 \leq (\sum_{i=1}^{n}a_i^2)(\sum_{i=1}^{n}b_i^2) $$   
\end{lemma}

\begin{proof}
    Tout d'abord si $\sum_{i=1}^{n}a_i^2 = 0 \implies a_i = 0(\forall i)$ \sep \\
    Donc l'inégalité est confirmée par $ (\sum_{i=1}^{n}0b_i)^2 \leq 0(\sum_{i=1}^{n}b_i^2) \implies 0 \leq 0$. \sep \\
    Supposons donc $\sum_{i=1}^{n}a_i^2 > 0$ \sep \\
    Soit $\sum_{i=1}^{n} (a_ix + b_i)^2$, Puisque $z^2 \geq 0(\forall z) \implies \sum_{i=1}^{n} (a_ix + b_i)^2 \geq 0 (\forall x)$ \sep \\
    Or $\sum_{i=1}^{n} (a_ix + b_i)^2 = \sum_{i=1}^{n} a_ix^2 + 2b_ia_ix + b_i^2$ \sep \\
    $= \sum_{i=1}^{n} a_i^2x^2 + \sum_{i=1}^{n} 2b_ia_ix + \sum_{i=1}^{n} b_i^2$ \sep \\
    $= (\sum_{i=1}^{n} a_i^2) \cdot x^2 + (\sum_{i=1}^{n} b_ia_i) \cdot 2x + (\sum_{i=1}^{n} b_i^2)$ \sep \\
    $= Ax^2 + 2Bx + C$, en posant $A = (\sum_{i=1}^{n} a_i^2), B = (\sum_{i=1}^{n} b_ia_i), C = (\sum_{i=1}^{n} b_i^2)$ \sep \\
    $=Ax^2 + 2Bx + \frac{B^2}{A} - \frac{B^2}{A} + C$ $(A > 0)$ \sep \\
    $=A \cdot (x^2 + 2\frac{B}{A}x + \frac{B^2}{A^2}) + C - \frac{B^2}{A}$ \sep \\
    $=A \cdot (x + \frac{B}{A})^2 + \frac{AC - B^2}{A}$ \sep \\
    Or puisque $\sum_{i=1}^{n} (a_ix + b_i)^2 \geq 0 (\forall x)$, c'est particulièrement vrai pour $x = \frac{-B}{A}$ \sep \\
    $\implies A \cdot (\frac{-B}{A} + \frac{B}{A})^2 + \frac{AC - B^2}{A} \geq 0$ \sep \\
    $\implies A \cdot (A \cdot 0 + \frac{AC - B^2}{A}) \geq (A \cdot 0)$ \sep \\
    $\implies AC - B^2 \geq 0$ \sep \\
    $\implies AC \geq B^2 \implies (\sum_{i=1}^{n} a_i^2)(\sum_{i=1}^{n} b_i^2) \geq (\sum_{i=1}^{n} b_ia_i)^2$ \sep \\
    Ce qui conclut notre Inégalité.
\end{proof}


\chapter{Espaces normés}

\section{Norme}
\begin{definition}
    \label{def:norme}
\end{definition}

\subsection{Norme dans R}

\subsection{Norme dans Rn}

\subsection{Norme par le produit scalaire}

\section{Espace préhilbertien}

\chapter{Espaces métriques}
\section{Les métriques}
\begin{definition}[Métrique]
    \label{def:metrique}
    Soit $X$, un ensemble, une \textbf{métrique} sur $X$ est une fonction $d: X \times X \to \Bbb R_+$ tel que: \\
    $d(x, y) \geq 0$ \\
    $d(x, y) = 0 \iff x = y$ \\
    $d(x, y) = d(y, x)$ \\
    $d(x, y) \leq d(x, z) + d(z, y)$ \\
\end{definition}
\subsection{Métrique sur l'ensemble des fonctions continues}
Prenons par exemple la métrique suivante définie sur l'ensemble des fonctions continues sur $[0, 1]$:
$$ d(f, g) = \sup_{x \in [0,1]} |f(x) - g(x)| $$

\begin{proof}[1ere prop.]
    $|f(x) - g(x)| \geq 0$, car $|x| \geq 0$, donc $d(f, g) \geq 0$ \sep \\
    Donc $d(x, y) \geq 0$
\end{proof}

\begin{proof}[2ieme prop.]
    Supposons que $d(f, g) = 0$, alors $\sup_{x \in [0, 1]} |f(x) - g(x)| = 0$. \sep \\
    $\implies (\forall x \in [0, 1]) |f(x) - g(x)| = 0$, car si il existe un $x$ tel que $|f(x) - g(x)| > 0$, alors le $sup = x$ \sep \\
    $\implies (\forall x \in [0, 1]) f(x) = g(x)$ \sep \\
    Maintenant supposons que $f(x) = g(x)$ \sep \\
    $\implies \sup_{x \in [0, 1]} |f(x) - g(x)| = \sup_{x \in [0, 1]} |f(x) - f(x)| = \sup_{x \in [0, 1]} |0| = 0$ \sep \\
    Donc $d(x, y) = 0 \iff x = y$
\end{proof} 

\begin{proof}[3ieme prop.]
    $|x - y| = |y - x|$ \sep \\
    $\implies |f(x) - g(x)| = |g(x) - f(x)|$ \sep \\
    $\implies \sup_{x \in [0, 1]} |f(x) - g(x)| = \sup_{x \in [0, 1]} |g(x) - f(x)|$ \sep \\
    $\implies d(x, y) = d(y, x)$ 
\end{proof}

\begin{proof}[4ieme prop.]
    Soit $|f(x) - g(x)| = |f(x) - h(x) + h(x) - g(x)|$ \sep \\
    $|f(x) - h(x) + h(x) - g(x)| \leq |f(x) - h(x)| + |h(x) - g(x)|$ \sep \\
    et $|f(x) - h(x)| \leq \sup_{x \in [0, 1]} |f(x) - h(x)|$ \sep \\
    $\implies |f(x) - h(x)| + |h(x) - g(x)| \leq \sup_{x \in [0, 1]} |f(x) - h(x)| + \sup_{x \in [0, 1]}|h(x) - g(x)| $ \sep \\
    Donc: $\sup_{x \in [0, 1]} |f(x) - g(x)| \leq$ \sep \\
    $\sup_{x \in [0, 1]} (|f(x) - h(x)| + |h(x) - g(x)|) \leq$ \sep \\
    $\sup_{x \in [0, 1]}(\sup_{x \in [0, 1]} |f(x) - h(x)| + \sup_{x \in [0, 1]}|h(x) - g(x)|) =$
    $\sup_{x \in [0, 1]} |f(x) - h(x)| + \sup_{x \in [0, 1]}|h(x) - g(x)|$
\end{proof}
\section{Convergence des suites}
\begin{definition}[Convergence]
    \label{def:convergence_metrique}
    Soit $(X, d)$, un espace métrique et ${x_n}$ une suite d'éléments de X. ${x_n}$ \textbf{converge} vers $x$ ($x_n \to x)$ si:
    $$ (\forall \epsilon > 0)(\exists M \in \Bbb N)\text{ tq }n > M \implies d(x_m, x) < \epsilon $$
\end{definition}

\begin{theorem}[Unicité convergence]
    Soit $(X, d)$ un espace métrique, $x, y \in X$ et $\{x_n\}$ une suite. Si $x_m \to x$ et $x_m \to y$, alors $x = y$
\end{theorem}

\begin{proof}
    Supposons que $x_n \to x$ et $x_n \to y$, mais que $x \neq y$. \sep \\
    Puisque $x \neq y \implies d(x, y) > 0$ \sep \\
    et $(\forall \epsilon > 0)(\exists M_1 \in \Bbb N, \exists M_2 \in \Bbb N)\text{ tq }d(x_{M1}, x) < \epsilon \land d(x_{M2}, y) < \epsilon$ 
    Posons $\epsilon = \frac{1}{2}d(x, y)$ et $M = \max{\{M_1,M_2\}}$ \sep \\
    Donc il existe un $m > M \implies d(x, x_m) < \frac{1}{2}d(x, y) \land d(y, x_m) < \frac{1}{2}d(x, y)$ \sep \\
    $\implies m > M \implies d(x, x_m) + d(y, x_m) \leq \frac{1}{2}d(x, y) + \frac{1}{2}d(x, y)$ \sep \\
    $\implies d(x, x_m) + d(x_m, y) \leq d(x, y)$ \sep \\
    Or $d(x, y) \leq d(x, x_m) + d(x_m, y)$, donc nous avons une contradiction.
\end{proof}
\section{Bornes}

\section{Suites de Cauchy}
\begin{definition}[Suite de Cauchy]
    \label{def:suite_cauchy}
    Soit $\{x_m\}$ une suite dans un espace métrique $(X, d)$. $\{x_m\}$ est une \textbf{suite de Cauchy} si:
    $$ \forall \epsilon > 0, \exists M \text{ tel que} m, n \geq M \implies d(x_m, x_n) < \epsilon $$
\end{definition}

\section{Complétude}
\begin{definition}[Completude]
    \label{def:completude}
    Soit $(X, d)$ un espace métrique. $(X, d)$ est dit \textbf{complet} si toute \hyperref[def:suite_cauchy]{suite de Cauchy} dans $X$ converge vers un élément de $X$.
\end{definition}

\section{Continuité}
\begin{definition}[Continuité]
    \label{def:continuite}
    Soit $(X, d_X)$ et $(Y, d_Y)$, deux espaces \hyperref[def:metrique]{métriques} et $f: X \to Y$, une fonction de $X$ vers $Y$.
    $f$ est \textbf{continue} en un point $x_0 \in X$ si:
    $$ \forall \epsilon > 0, \exists \delta > 0,\text{ tel que }d_X(x, x_0) < \delta \implies d_Y(f(x), f(x_0)) < \epsilon $$ 
\end{definition}

On observe un lien entre la convergence et la continuité:
\begin{theorem}
    \label{thm:continuite_convergence} Soit $(X, d_X)$, $(Y, d_Y)$ et $f:X \to Y$. \\
    $f$ est  \hyperref[def:continuite]{continue} au point 
    $a \in X$ $\iff$ pour toute suite $\{x_n\}$ tel que $x_n \to a$, la suite $\{f(x_n)\}$ \hyperref[def:convergence_metrique]{converge} vers $f(a)$.
\end{theorem}

\begin{proof}[=>]
    Soit $f$, continue en $a$ et $x_n \to a$ \sep \\
    Alors, $\forall \epsilon_1 > 0, \exists M : m \geq M \implies d(x_m, a) < \epsilon_1$ \sep
    et, $\forall \epsilon > 0, \exists \delta > 0, d_X(x_n, a) < \delta \implies d_Y(f(x_n), f(a)) < \epsilon$ \sep \\
    En posant $\delta = \epsilon_1$ \sep \\
    On obtient $\forall \epsilon > 0, \exists M : m \geq M \implies d(x_m, a) < \epsilon_1 \implies d(f(x_m), f(a)) < \epsilon$ \sep \\
    et en réduistant $\forall \epsilon > 0, \exists M : m \geq M \implies d(f(x_m), f(a)) < \epsilon$ \sep \\
\end{proof}

\begin{comment}
    Nos hypothèses de départ. \sep
    En utilisant la définition d'une \hyperref[def:convergence_metrique]{suite convergente}. \sep
    En utilisant la définition de la \hyperref[def:continuite]{continuité} au point $a$ \sep
    Puisque $x_n$ converge vers $a$, on peut substituer $\epsilon_1$ pour $\delta$ \sep.
    On conclut
\end{comment}

\begin{proof}[<=]
    Soit $x_n \to a$ et $f(x_n) \to f(a)$ \sep \\
    $\forall \epsilon > 0, \exists M_1 : m \geq M_1 \implies d(x_m, a) < \epsilon$ et $\forall \epsilon > 0, \exists M_2 : m \geq M_2 \implies d(f(x_m), f(a)) < \epsilon$ \sep \\
    Posons $M =\max{M_1, M_2}$ \sep \\
    Donc
\end{proof}

\section{Contractions}
\begin{definition}[Lambda-contractante]
    \label{def:lambda_contractante}
    Soit $f: X \to Y$, $f$ est \textbf{$\lambda$-contractante} si 
    $$ \exists \lambda, \forall x,y \in X, d_Y(f(x), f(y)) \leq \lambda d_X(x,y) $$
\end{definition}

\begin{definition}[Contraction]
    \label{def:contraction}
    Soit $f: X \to Y$, $f$ est une \textbf{contraction} si elle est \hyperref[def:lambda_contractante]{$\lambda$-contractante}
    pour $$ 0 \leq \lambda \leq 1$$
\end{definition}

\begin{lemma}
    \label{lem:contractante_continue}
    Si $f: X \to X$ est \hyperref[def:contraction]{contractante}, alors $f$ est \hyperref[def:continuite]{continue}
\end{lemma}
\begin{proof}
    Soit $f: X \to X$, $\lambda$-contractante \sep 
    Alors $d(f(x), f(y)) \leq \lambda d(x, y)$ \sep. On cherche à démontrer que 
    $\forall \epsilon > 0, \exists \delta > 0, d(x, y) < \delta \implies d(f(x), f(y)) < \epsilon$ \sep 
    En posant $\delta = \frac{\epsilon}{\lambda + 1}$ ($\lambda$ peut $= 0$) \sep 
    $\delta > 0$, car $\epsilon > 0$ Donc si $d(x, y) < \frac{\epsilon}{\lambda + 1}$ et que $d(f(x), f(y)) \leq \lambda d(x,y)$ \sep 
    alors $d(f(x), f(y)) \leq \lambda d(x,y) < \lambda \cdot \frac{\epsilon}{\lambda + 1}$ \sep 
    Or $\lambda \cdot \frac{\epsilon}{\lambda + 1} <\epsilon$ \sep 
    En effet, si $\lambda = 0 \implies 0 < \epsilon$ \sep 
    Si $0 < \lambda$, $\frac{\epsilon}{\lambda + 1} < \frac{\epsilon}{\lambda} \implies \lambda \cdot \frac{\epsilon}{\lambda + 1} < \lambda \cdot \frac{\epsilon}{\lambda}$ \sep
    Donc $\forall \epsilon > 0, \exists \delta = \frac{\epsilon}{\lambda + 1}, d(x, y) < \delta \implies d(f(x), f(y)) < \lambda \cdot \frac{\epsilon}{\lambda + 1} < \epsilon$  \sep     
\end{proof}

\section{Points fixes}

\begin{definition}[Point fixe]
    \label{def:point_fixe}
    Soit $f: X \to X$ et $x_0 \in X$, $x_0$ est un \textbf{point fixe} de $f$ si, $f(x_0) = x_0$.
\end{definition}

\begin{theorem}[Point fixe de Banach]
    \label{thm:point_fixe_banach}
    Soit $f: X \to X$, une \hyperref[def:contraction]{contraction}, si X est un espace métrique \hyperref[def:completude]{complet},
    $f$ admet un \hyperref[def:point_fixe]{point fixe} unique.
\end{theorem}

\begin{proof}[I. Suite de Cauchy récursive]
    Soit $\{x_n\}$, une suite d'éléments de $X$ tel que: $\{x_1 = f(x_0), x_2 = f(x_1) = f(f(x_0)), x_3 = f(x_2), \dots \}$ \sep
    Démontrons que cette suite $\{x_n\}$ est une suite de Cauchy: \sep \\
    Pour un $n > 1$, $d(f(x_{n-1}), f(x_n)) \leq \lambda d(x_{n-1}, x_n)$ \sep
    Or $d(f(x_{n-1}), f(x_n)) = d(x_n x_{n+1})$ \sep 
    Donc $d(x_n x_{n+1}) \leq \lambda d(x_{n-1}, x_n) \leq \lambda \cdot \lambda d(x_{n-2}, x_{n-1}) \leq \dots $ \sep
    Ou en généralisant $d(x_n x_{n+1}) \leq \lambda^n d(x_0, x_1)$ \sep 
    Or $d(x_n, x_m) \leq d(x_n, x_{n+1}) + d(x_{n+1}, x_{n+2}) + \dots + d(x_{m-1}, x_m)$ \sep 
    Donc $d(x_n, x_m) \leq  \lambda^n d(x_0, x_1) + \lambda^{n+1} d(x_0, x_1) + \dots + \lambda^{m-1} d(x_0, x_1)$ \sep
    En factorisant $d(x_n, x_m) \leq \lambda^n \cdot d(x_0, x_1) \cdot (1 + \lambda + \lambda^2 + \dots + \lambda^{m-n-1})$ \sep 
    $(1 + \lambda + \lambda^2 + \dots + \lambda^{m-n-1}) = \sum_{i=0}^{m-n-1} \lambda^i$  \sep 
    $\sum_{i=0}^{m-n-1} \lambda^i < \sum_{i=0}^{\infty} \lambda^i$ \sep 
    Puisque $0 \leq \lambda < 1$, $\sum_{i=0}^{m-n-1} \lambda^i < \frac{1}{1-\lambda}$ \sep
    Donc nous avons $d(x_n, x_m) \leq \lambda^n \cdot d(x_0, x_1) \cdot \sum_{i=0}^{m-n-1} \lambda^i <  \lambda^n \cdot d(x_0, x_1) \cdot \frac{1}{1-\lambda} $ \sep
    Posons: $d(x_0, x_1) = \alpha$, car c'est une valeur réelle finie \sep 
    Si $\alpha = 0$ ou $\lambda = 0$, la solution est triviale. \sep 
    De plus $\lambda^{\lceil n \rceil} \cdot d(x_0, x_1) \cdot \frac{1}{1-\lambda} < \lambda^{n} \cdot d(x_0, x_1) \cdot \frac{1}{1-\lambda}$, car $\lambda < 1$ \sep
    Posons $n = \log_{\lambda}\frac{\epsilon(1 - \lambda)}{\alpha}$ \sep 
    Ainsi $\lambda^{\lceil n \rceil} \cdot \alpha \cdot \frac{1}{1-\lambda} < \lambda^{\log_{\lambda}\frac{\epsilon(1 - \lambda)}{\alpha}} \cdot \alpha \cdot \frac{1}{1-\lambda}$ \sep
    $\implies \lambda^{\lceil n \rceil} \cdot \alpha \cdot \frac{1}{1-\lambda} < \frac{\epsilon(1 - \lambda)}{\alpha} \cdot \frac{\alpha}{(1 - \lambda)}$ \sep 
    $\implies \lambda^{\lceil n \rceil} \cdot \alpha \cdot \frac{1}{1-\lambda} < \epsilon$ \sep
    Donc c'est une suite de Cauchy.
\end{proof}

\begin{comment}
    On construit une suite récursive, qui on le verra aura des propriétés intéressantes. \sep 
    On cherche à démontrer que c'est une suite de Cauchy. \sep
    Puisque $f$ est une fonction \hyperref[def:lambda_contractante]{$\lambda$-contractante} \sep 
    On utilise la définition de notre suite récursive. \sep 
    On "remonte" la différence de la suite jusqu'au début. \sep 
    Attention ca ne marche que pour $\lambda \geq 0$!. \sep 
    Par l'inégalité du triangle sur les suites. \sep 
    Grâce à notre inégalité $d(x_n x_{n+1}) \leq \lambda^n d(x_0, x_1)$ \sep
    On factorise nos $\lambda$ \sep 
    On équivaut nos $\lambda$ en une sommation pour retrouver la série géométrique \sep.
    En effet, une série finie positive est nécessairement plus petite qu'une série infinie positive \sep.
    On utilise le fait que la série géométrique infinie converge vers $\frac{1}{1-\lambda}$ \sep.
\end{comment}

\begin{proof}[II. Existence du point fixe]
    Soit $\{x_n\}$, une suite de Cauchy dans $X$ tel que $\{x_1 = f(x_0), x_2 = f(x_1) = f(f(x_0)), x_3 = f(x_2), \dots \}$ \sep
    Puisque $X$ est complet, $\{x_n\}$ converge vers un point $x \in X$ \sep 
    La sous-suite $\{x_{(n+1)}\}$ converge également vers le point $x \in X$ \sep
    Et $\{f(x_n)\} = \{x_{(n + 1)}\}$ \sep 
    Or puisque $f$ est continue, $\{f(x_n)\}$ converge vers $f(x)$ \sep 
    Donc $\{f(x_n)\}$ converge vers $f(x)$ et converge vers $x$ \sep
    Mais par le théorême d'unicité, $f(x) = x$. Un point fixe \sep  
\end{proof}
\begin{proof}[III. Unicité du point fixe]
    Supposons qu'il existe deux points fixes tel que $x_0$, $x_1$ et $f(x_0) = x_0$ et $f(x_1) = x_1$. \sep 
    Puisque $f$ est une contraction $d(f(x_0), f(x_1)) \leq \lambda d(x_0, x_1)$ \sep 
    Et puisque $f(x_0) = x_0$ et $f(x_1) = x_1$, $\implies d(x_0, x_1) \leq \lambda d(x_0, x_1)$ \sep 
    $\implies d(x_0, x_1) \leq 0 \implies x_0 = x_1$
\end{proof}

\chapter{Espaces topologiques}
\section{Topologie}
\begin{definition}[Topologie]
    \label{def:topologie} 
    Soit $\mathcal{O}$, une \hyperref[def:famille]{famille} est une \textbf{topologie} sur $X$ si et seulement si:\\
    $(x \in \mathcal{O} \implies x \subset X)$ (chaque élément de la famille est un sous-ensemble de $X$)\\
    $(\emptyset \in \mathcal{O} \land X \in \mathcal{O})$ \\
    $(x \subset \mathcal{O} \land y \subset \mathcal{O} \implies x \cap y \in \mathcal{O})$ (si $x,y$ sont finis) \\
    $(x \subset \mathcal{O} \land y \subset \mathcal{O} \implies x \cup y \in \mathcal{O})$ \\
\end{definition}
En d'autres mots, c'est la famille de tous les ensembles qu'on peut former de $X$ lesquels peuvent se réunir ou s'intersecter et 
générer un ensemble qui fait partie de la famille.
\section{Fermeture}
\section{Continuité}


\appendix
\chapter{Notions préalables}
Voici une série de définition importantes à considérer lors de la lecture de cet ouvrage. Celles-ci servent
de références pour les exemples et les preuves qui seront amenés dans les chapitres suivants. Pour une explication
plus complète et rigoureuse, il sera nécessaire de se réferrer aux livre précédent: Analyse des Réels.

\begin{definition}[Voisinage dans les Réels]
    \label{def:voisinage_reels}
    Ensemble de point satisfaisant l'expression
    suivante: $$V(a, \delta) = \{ x \in \Bbb R : |x - a| < \delta \}$$
\end{definition}

\begin{definition}[Voisinage troué dans les Réels]
    \label{def:voisinage_troue_reels}
    Ensemble de point satisfaisant l'expression
    suivante: $$V'(a, \delta) = \{ x \in \Bbb R : |x - a| < \delta \land x \neq a \}$$
\end{definition}

\begin{definition}[Points intérieurs de E]
    \label{def:point_int}
    Ensemble de points satisfaisant l'expression
    suivante: $$int(E) = \{ x \in E : \exists \delta > 0, V(x, \delta) \subseteq E \} $$
\end{definition}

\begin{definition}[Ensemble ouvert]
    \label{def:ensemble_ouvert}
    Un ensemble est dit ouvert si tous ces éléments sont des \hyperref[def:point_int]{points intérieurs.}
\end{definition}

\begin{definition}[Groupe abélien]
    \label{def:groupe_abelien}
    Un \textbf{groupe abélien} ou \textbf{groupe non-commutatif} est un groupe
    (une structure algébrique associative avec un élément neutre et un élément inverse $(\forall a \in G, \exists a^{-1})$), 
    dont l'opération binaire est commutative.
\end{definition}

\begin{definition}[Famille d'ensemble]
    \label{def:famille}
    Une \textbf{famille} ou \textbf{collection} est simplement un ensemble d'ensemble.
\end{definition}
\todo{Add reference}

\chapter{Cheatsheet}
\section{Formules}
\subsection{Inégalité de Cauchy-Schwarz}
$$(\sum_{i=1}^{n} a_ib_i)^2 \leq \sum_{i=1}^{n} a_i^2 \cdot \sum_{i=1}^{n} b_i ^ 2$$ 
\subsection{Convergence série géométrique}
$$|r| < 1 \implies \sum_{n=0}^{\infty} r^n \to \frac{1}{1 - r} $$
\subsection{Inégalité du triangle}
$$| ||x|| - ||y|| | \leq ||x \pm y|| \leq ||x|| + ||y|| $$
\section{Types d'espace}
\begin{itemize}
    \item Espace vectoriel: Ensemble d'éléments appellés vecteurs, muni de l'addition et de la multiplication par un scalaire.
    \item Espace préhilbertien: Espace vectoriel muni d'un produit scalaire.
    \item Espace normé: Espace vectoriel muni d'une norme.
    \item Espace métrique: Ensemble \textbf{quelconque} munie d'une métrique.
    \item Espace métrique complet:
    \item Espace topologique:
\end{itemize}
\section{Fonctions spéciales}
\subsection{Produit scalaire}
\begin{itemize}
    \item Un produit scalaire est une fonction de $V \times V \to \Bbb R$, noté $\langle x, y \rangle$ avec 5 propriétés qui la définisse
    \item $\langle x, x \rangle \geq 0$
    \item $\langle x, x \rangle = 0 \iff x = 0$
    \item $\langle x, y \rangle = \langle y, x \rangle$
    \item $\langle x + y, z\rangle = \langle x, z \rangle + \langle y, z \rangle$
    \item $\alpha\langle x, y \rangle = \langle \alpha x, y \rangle \forall \alpha \in \Bbb R$
\end{itemize}
\subsection{Norme}
\begin{itemize}
    \item Une norme sur $V$ est une fonction $V$ vers $\Bbb R$, noté $x \mapsto ||x||$ avec 4 propriétés qui la définisse:
    \item $||x|| \geq 0$
    \item $||x|| = 0 \iff x = 0$
    \item $||\alpha x|| = |\alpha| \cdot ||x||$
    \item $||x + y || \leq ||x|| + ||y||$
    \item Norme euclidienne: $\sqrt{x_1^2 + x_2^2 + \dots + x_n^2}$
    \item Le produit scalaire est une norme.
\end{itemize}
\subsection{Métrique}
\begin{itemize}
    \item Une métrique est une fonction $X \times X \to 0, \infty$ avec les 4 propriétés suivantes:
    \item $d(x,y) \geq 0$
    \item $d(x,y) = d(y, x)$
    \item $d(x,y) = 0 \iff x = y$
    \item $d(x,y) \leq d(x, z) + d(z, y)$
    \item Métrique discrète: $d(x,y) = 0$ si $x = y$, 1 si $x \neq 1$
    \item La norme est une métrique
\end{itemize}
\section{Définition topologiques}
\begin{itemize}
    \item \textbf{Boule ouverte} : $B(x, r) = \{y \in R^n : || x - y || < r\}, r > 0$
    \item \textbf{Boule fermée} : $B(x, r) = \{y \in R^n : || x - y || \leq r\}, r > 0$
    \item \textbf{Point intérieur}: $x \in \Bbb R^n \text{ est un point intérieur de } E \iff \exists \delta > 0 : B(x, \delta) \subset E$
    \item \textbf{Ensemble ouvert}: $E\text{ est ouvert} \iff E = int(E)$
    \item \textbf{Convergence suite}: $\forall \epsilon > 0, \exists M \in \Bbb N \text{ t q} m \geq M \implies d(x_m, x) < \epsilon$
    \item \textbf{Ensemble borné}: $\exists x_0 \in X \land \exists r > 0, \forall x \in A, d(x_0, x) < r$
\end{itemize}

\end{document}