\documentclass[12pt]{book}

\usepackage[utf8]{inputenc} % Required for inputting international characters
\usepackage[T1]{fontenc} % Output font encoding for international characters

\usepackage{comment}
\usepackage{mathpazo} % Palatino font
\usepackage{amssymb}
\usepackage{amsmath}
\usepackage{array}
\usepackage{amsthm}
\usepackage{graphicx}
\usepackage{multirow}
\usepackage{relsize}
\usepackage{hyperref}
\usepackage{xcolor}
\hypersetup{
    colorlinks,
    linkcolor={red!50!black},
    citecolor={blue!50!black},
    urlcolor={blue!80!black}
}
\usepackage[makeroom]{cancel}

\usepackage[margin=0.5in]{geometry}
\newcommand{\overbar}[1]{\mkern 1.5mu\overline{\mkern-1.5mu#1\mkern-1.5mu}\mkern 1.5mu}

\newcommand*\conj[1]{\overbar{#1}}
% STATS Shortcut
\newcommand{\BP}{\Bbb P}
\newcommand{\BR}{\Bbb R}
\let\Bbb\mathbb
\def\sep{\phantom{}}
\newcommand\todo[1]{\phantom{#1}}
\theoremstyle{definition}
\newtheorem{definition}{Définition}[section]
\newtheorem*{example}{Exemple}
\newtheorem{theorem}{Theorême}[section]
\newtheorem{corollary}{Corollaire}[theorem]
\newtheorem{lemma}[theorem]{Lemme}  

\title{Topologie}

\begin{document}
\chapter{Introduction}
Qu'entend-on par espaces vectoriels?

\chapter{Compacité}

\chapter{Continuité}
\begin{definition}
    \label{def:continuite}
    Soit $f$, une fonction de domaine $\Bbb R^n$. $f$ est continue en $a$ si 
    $$\forall \epsilon > 0, \exists \delta > 0 \text{ tq } x \in D \land 0 < ||x - a|| < \delta \implies || f(x) - f(a) || < \epsilon$$
\end{definition}

Si une fonction est continue sur l'entiereté de son domaine, on dira alors que celle-ci est continue uniformément:
\begin{definition}
    \label{def:continuite_uniforme}
    Soit $f$, une fonction de domaine $\Bbb R^n$. $f$ est \textbf{uniformément continue} sur $D$ si 
    $$\forall \epsilon > 0, \exists \delta > 0 \text{ tq } \forall x, a \in D \land 0 < ||x - a|| < \delta \implies || f(x) - f(a) || < \epsilon$$
\end{definition}

\begin{theorem}
    \label{def:compact_implique_continuite_uniforme}
    Soit $f$, une fonction continue de domaine $K$ et $K \subseteq \Bbb R^n$. Si $K$ est \hyperref[def:compacite]{compact}, alors $f$ est 
    \hyperref[def:continuite_uniforme]{uniformément continue}.
\end{theorem}

\chapter{Notions de limite}
\begin{definition}
    \label{def:limite}
    Soit $f$, une fonction de domaine $\Bbb R^n$. Le point $b$ est considéré la \textbf{limite} de $f$ à un point d'accumulation $c$ du domaine si
    $$\forall \epsilon > 0, \exists \delta > 0 \text{ tq } x \in D \land 0 < ||x - c|| < \delta \implies || f(x) - b || < \epsilon$$
\end{definition}

\begin{theorem}[Unicité de la limite]
    \label{thm:limite_unique}
    La limite, si elle existe est unique.
\end{theorem}

\chapter{Connexité}

\begin{theorem}
    Si $f: K \to \Bbb R$ est continue et $K \subseteq \Bbb R^m$, $K$ compact. Alors $\exists f(a) = inf\{f(x) : x \in K\}, f(b) = inf\{f(x) : x \in K\}$.
\end{theorem}

\begin{proof}
    Puisque $K$ est compact et $f$, continue, alors $f(K)$ est compact.\\
    $f(K)$ a donc une borne tq $-M \leq f(K) \leq M$.\\
    Par conséquent, il existe un $m = sup\{f(x): x \in K\}$ et $m' = inf\{f(x): x \in K\}$\\
    Par définition du sup et inf, $m$ et  $m'$ sont des points d'accumulations de $f(K)$.\\
    Puisque $f(K)$ est fermé, $m$ et $m'$ $\in f(K)$. Donc il existe un point $a = f(a) = m$.
\end{proof}
\appendix
\chapter{Notions préalables}
Voici une série de définition importantes à considérer lors de la lecture de cet ouvrage. Celles-ci servent
de références pour les exemples et les preuves qui seront amenés dans les chapitres suivants. Pour une explication
plus complète et rigoureuse, il sera nécessaire de se réferrer aux livre précédent: Analyse des Réels.

\begin{definition}[Voisinage dans les Réels]
    \label{def:voisinage_reels}
    Ensemble de point satisfaisant l'expression
    suivante: $$V(a, \delta) = \{ x \in \Bbb R : |x - a| < \delta \}$$
\end{definition}

\begin{definition}[Voisinage troué dans les Réels]
    \label{def:voisinage_troue_reels}
    Ensemble de point satisfaisant l'expression
    suivante: $$V'(a, \delta) = \{ x \in \Bbb R : |x - a| < \delta \land x \neq a \}$$
\end{definition}

\begin{definition}[Points intérieurs de E]
    \label{def:point_int}
    Ensemble de points satisfaisant l'expression
    suivante: $$int(E) = \{ x \in E : \exists \delta > 0, V(x, \delta) \subseteq E \} $$
\end{definition}

\begin{definition}[Ensemble ouvert]
    \label{def:ensemble_ouvert}
    Un ensemble est dit ouvert si tous ces éléments sont des \hyperref[def:point_int]{points intérieurs.}
\end{definition}

\begin{definition}[Groupe abélien]
    \label{def:groupe_abelien}
    Un \textbf{groupe abélien} ou \textbf{groupe non-commutatif} est un groupe
    (une structure algébrique associative avec un élément neutre et un élément inverse $(\forall a \in G, \exists a^{-1})$), 
    dont l'opération binaire est commutative.
\end{definition}

\begin{definition}[Famille d'ensemble]
    \label{def:famille}
    Une \textbf{famille} ou \textbf{collection} est simplement un ensemble d'ensemble.
\end{definition}

\begin{definition}[Compacité]
    
\end{definition}

\end{document}