\documentclass[12pt]{book}

\usepackage[utf8]{inputenc} % Required for inputting international characters
\usepackage[T1]{fontenc} % Output font encoding for international characters

\usepackage{comment}
\usepackage{mathpazo} % Palatino font
\usepackage{amssymb}
\usepackage{amsmath}
\usepackage{array}
\usepackage{amsthm}
\usepackage{graphicx}
\usepackage{multirow}
\usepackage{relsize}
\usepackage{hyperref}
\usepackage{xcolor}
\hypersetup{
    colorlinks,
    linkcolor={red!50!black},
    citecolor={blue!50!black},
    urlcolor={blue!80!black}
}
\usepackage[makeroom]{cancel}

\usepackage[margin=0.5in]{geometry}
\newcommand{\overbar}[1]{\mkern 1.5mu\overline{\mkern-1.5mu#1\mkern-1.5mu}\mkern 1.5mu}

\newcommand*\conj[1]{\overbar{#1}}
% STATS Shortcut
\newcommand{\BP}{\Bbb P}
\newcommand{\BR}{\Bbb R}
\let\Bbb\mathbb
\def\sep{\phantom{}}
\theoremstyle{definition}
\newtheorem{definition}{Définition}[section]
\newtheorem*{example}{Exemple}
\newtheorem{theorem}{Theorême}[section]
\newtheorem{corollary}{Corollaire}[theorem]
\newtheorem{lemma}[theorem]{Lemme}  

\title{Modèles de régression}

\begin{document}
\chapter{Introduction}
Comment modéliser une relation statistique? Existe-il un lien entre deux ou plusieurs variables, et si oui, comment
le représenter. Il est possible lors de la création d'un modèle de régression de modéliser ce type de relation. Un modèle de régression tente de créer un lien mathématique entre ces 
variables et permet même d'estimer à un certain degré de certitude une prévision quant à la valeur que prendra ces 
variables dans certains contextes.

Par exemple, existe-t-il un lien entre le coût d'une main-d'oeuvre et le nombre d'unités produites? Grâce à un modèle de régression
linéaire, on peut décrire et établir ce lien.

\chapter{Régression linéaire simple}
\subsection{Diagramme de dispersion}
\subsection{Description du modèle}
Soit X, une variable aléatoire indépendante et Y une variable aléatoire dépendante. Supposant un lien entre cette variable X et Y 
et la présence de $n$ couples de données $(x_i, y_i), i=1 \dots n$, le modèle linéaire de régression linéaire simple aura la forme 
suivante:
$$ y_i = \beta_0 + \beta_1 x_i + \epsilon_i, i=1 \dots n $$ 


Ce modèle est caractérisé par trois paramètres: $\beta_0, \beta_1 \text{ et } \epsilon$.

\begin{itemize}
    \item $\beta_0$ est une constante.
    \item $\beta_1$ est un coefficient.
    \item $\epsilon$ est une variable aléatoire qui varie selon l'index $i$ et représente le degré d'erreur du modèle en soit.
\end{itemize}

\begin{example}
    Supposons que l'on cherche d'établir combien nous coûterait la production d'1 million de canards en plastique.
    On a X, le nombre de canards en plastique produit. \sep
    On a Y, le coût en main d'oeuvre. \sep
    On a présentement $n=5$ données basées sur nos observations dans notre usine. \sep
    Supposons qu'on aie $(x_1 = 500, y_1 = 1000\$)$,$ (x_2 = 10, y_2 = 500\$)$, \dots, $(x_5 = 5000, y_5 = 50400\$)$ \sep 
    À partir de ce moment on peut trouver un $\beta_0, \beta_1 $ et établir l'intervalle de $\epsilon$. \sep
    Dans cet exemple, $\beta_0 = 400$, $\beta_1 = 10$ \sep 
    et $\epsilon = 0$ et ne varie pas. En effet, les valeurs arrivent \textit{exactement} et n'ont donc pas d'erreur.
\end{example}
Concrètement parlant, si on a une variable X représentant le nombre d'unitées produites et une variable Y représantant le coût de la main d'oeuvre pour produire ces unités,
on aurait $x_1$ qui représente la première valeur
Un modèle de régression linéaire simple représente le lien entre une variable indépendante X [TODO ADD DEF] et une variable dépendante Y.
Notez bien l'utilisation

\end{document}