\documentclass[12pt]{book}

\usepackage[utf8]{inputenc} % Required for inputting international characters
\usepackage[T1]{fontenc} % Output font encoding for international characters

\usepackage{comment}
\usepackage{mathpazo} % Palatino font
\usepackage{amssymb}
\usepackage{amsmath}
\usepackage{array}
\usepackage{amsthm}
\usepackage{graphicx}
\usepackage{multirow}
\usepackage{relsize}
\usepackage{hyperref}
\usepackage{xcolor}
\usepackage{xr-hyper} % Required for cross referencing

% EXTERNAL REFERENCES
\externaldocument[topologie/]{../topologie/book}
% ----

\hypersetup{
    colorlinks,
    linkcolor={red!50!black},
    citecolor={blue!50!black},
    urlcolor={blue!80!black}
}
\usepackage[makeroom]{cancel}

\usepackage[margin=0.5in]{geometry}
\newcommand{\overbar}[1]{\mkern 1.5mu\overline{\mkern-1.5mu#1\mkern-1.5mu}\mkern 1.5mu}

\newcommand*\conj[1]{\overbar{#1}}
% STATS Shortcut
\newcommand{\BP}{\Bbb P}
\newcommand{\BR}{\Bbb R}
\let\Bbb\mathbb
\def\sep{\phantom{}}
\newcommand\todo[1]{\phantom{#1}}
\theoremstyle{definition}
\newtheorem{definition}{Définition}[section]
\newtheorem*{example}{Exemple}
\newtheorem{theorem}{Theorême}[section]
\newtheorem{corollary}{Corollaire}[theorem]
\newtheorem{lemma}[theorem]{Lemme}  

\title{Espaces Vectoriels}


\begin{document}
\chapter{Introduction}
Qu'entend-on par espaces vectoriels?

\chapter{Compacité}
Tel que vu dans les espaces métriques, un sous-ensemble est compact implique qu'il est borné et fermé.
L'inverse n'est toutefois pas vrai. Oh la la \hyperref[topologie/def:convergence_metrique]{C'est beau}

Néanmoins, dans un espace vectoriel, nous obtenons la double-implication suivante:
\begin{theorem}[Heine-Borel]
    \label{thm:heine_borel} Un sous-ensemble de $\Bbb R^n$ est \hyperref[def:compacite]{compact} si et seulement s'il est 
    borné et fermé.
\end{theorem}.

Il est donc entierement valable de définir la compacité comme une équivalence complète avec le fait d'être borné et fermé tant 
qu'on est dans un espace vectoriel.

\begin{theorem}
    \label{thm:compact_implique_compact}
    Soit $f$ de domaine $D \subseteq \Bbb R^n$, Si $D$ est compact et $f$ continue, alors $f(D)$ est compact.
\end{theorem}

\begin{proof}
\end{proof}

\chapter{Continuité}
\begin{definition}
    \label{def:continuite}
    Soit $f$, une fonction de domaine $D \subseteq \Bbb R^n$. $f$ est continue en $a$ si 
    $$\forall \epsilon > 0, \forall x, \exists \delta > 0, \forall a \text{ tq } x \in D \land 0 < ||x - a|| < \delta \implies || f(x) - f(a) || < \epsilon$$
\end{definition}

Il existe aussi la notion de continuité uniforme. Celle-ci implique que l'on peut trouver un ration $\delta-\epsilon$ minimal qui couvre
l'entiereté des points.
\begin{definition}
    \label{def:continuite_uniforme}
    Soit $f$, une fonction de domaine $\Bbb R^n$. $f$ est \textbf{uniformément continue} sur $D$ si 
    $$\forall \epsilon > 0, \exists \delta > 0 \text{ tq } \forall x, a \in D \land 0 < ||x - a|| < \delta \implies || f(x) - f(a) || < \epsilon$$
\end{definition}

\begin{theorem}
    \label{def:compact_implique_continuite_uniforme}
    Soit $f$, une fonction continue de domaine $K$ et $K \subseteq \Bbb R^n$. Si $K$ est \hyperref[def:compacite]{compact}, alors $f$ est 
    \hyperref[def:continuite_uniforme]{uniformément continue}.
\end{theorem}

\begin{proof}
    Supposons $f$ n'est pas uniformément continue lorsque $K$ compact et $f$ continue. \\\sep
    Ainsi il existerait un $\epsilon_0$ qui pour un certain $x,a$ implique que la norme $||f(x) - f(a)|| \geq \epsilon_0$ et ce pour tout $\delta$\\\sep 
    Posons $\delta = \frac{1}{n}$, ainsi il existe un $x_n, a_n$ tq $||x_n - a_n|| < \frac{1}{n} \implies ||f(x_n) - f(a_n)|| \geq \epsilon_0$\\\sep
    Puisque c'est vrai pour chaque $\delta$, c'est vrai pour chaque $n \in \Bbb N$, nous avons donc une suite $x_n$\\\sep
    De plus ${||x_n - a_n||} \to 0$, puisque la suite $\frac{1}{n} \to 0$\\\sep 
    Toutefois, puisque $K$ est compact, il existe une sous-suite de $x_{n_l}$ qui converge vers un point $x_0$ \\\sep
    $\forall \epsilon, \exists M \text{ tq } {n_l} \geq M \implies ||x_{n_l} - x_0|| < \epsilon$ \\\sep
    Puisque $||a_{n_l} - x_0|| \leq ||a_{n_l} - x_{n_l}|| + ||x_{n_l} - x_0 ||$, il en suit que ${a_{n_l}} \to 0$ aussi \\\sep
    Vu que $f$ est continue, $\{f(a_{n_l})\} \to f(x_0)$ et $\{f(x_{n_l})\} \to f(x_0)$ \\\sep 
    Alors $||f(x_{n_l}) - f(x_0)|| < \frac{\epsilon}{2} \forall \epsilon$ et $||f(x_0) - f(a_{n_l})|| < \frac{\epsilon}{2} \forall \epsilon$ \\\sep 
    Ainsi $||f(x_{n_l}) - f(a_{n_l})|| \leq |f(x_{n_l}) - f(x_0)|| + ||f(x_0) - f(a_{n_l})|| < \epsilon \forall \epsilon$ \\\sep 
    Une contradiction avec notre affirmation que $||f(x) - f(a)|| \geq \epsilon_0$
\end{proof}

\chapter{Notions de limite}
\begin{definition}
    \label{def:limite}
    Soit $f$, une fonction de domaine $\Bbb R^n$. Le point $b$ est considéré la \textbf{limite} de $f$ à un point d'accumulation $c$ du domaine si
    $$\forall \epsilon > 0, \exists \delta > 0 \text{ tq } x \in D \land 0 < ||x - c|| < \delta \implies || f(x) - b || < \epsilon$$
\end{definition}

\begin{theorem}[Unicité de la limite]
    \label{thm:limite_unique}
    La limite, si elle existe est unique.
\end{theorem}

\begin{theorem}
    Si $f: K \to \Bbb R$ est continue et $K \subseteq \Bbb R^m$, $K$ compact. Alors $\exists f(a) = inf\{f(x) : x \in K\}, f(b) = inf\{f(x) : x \in K\}$.
\end{theorem}


\begin{proof}
    Puisque $K$ est compact et $f$, continue, alors $f(K)$ est compact.\\
    $f(K)$ a donc une borne tq $-M \leq f(K) \leq M$.\\
    Par conséquent, il existe un $m = sup\{f(x): x \in K\}$ et $m' = inf\{f(x): x \in K\}$\\
    Par définition du sup et inf, $m$ et  $m'$ sont des points d'accumulations de $f(K)$.\\
    Puisque $f(K)$ est fermé, $m$ et $m'$ $\in f(K)$. Donc il existe un point $a = f(a) = m$.
\end{proof}

\chapter{Connexité}
\chapter{Fonctions linéaires}
\chapter{Dérivation}
\chapter{Dérivation partielle}
\appendix
\chapter{Notions préalables}
Voici une série de définition importantes à considérer lors de la lecture de cet ouvrage. Celles-ci servent
de références pour les exemples et les preuves qui seront amenés dans les chapitres suivants. Pour une explication
plus complète et rigoureuse, il sera nécessaire de se réferrer aux livre précédent: Analyse des Réels.

\begin{definition}[Voisinage dans les Réels]
    \label{def:voisinage_reels}
    Ensemble de point satisfaisant l'expression
    suivante: $$V(a, \delta) = \{ x \in \Bbb R : |x - a| < \delta \}$$
\end{definition}

\begin{definition}[Voisinage troué dans les Réels]
    \label{def:voisinage_troue_reels}
    Ensemble de point satisfaisant l'expression
    suivante: $$V'(a, \delta) = \{ x \in \Bbb R : |x - a| < \delta \land x \neq a \}$$
\end{definition}

\begin{definition}[Points intérieurs de E]
    \label{def:point_int}
    Ensemble de points satisfaisant l'expression
    suivante: $$int(E) = \{ x \in E : \exists \delta > 0, V(x, \delta) \subseteq E \} $$
\end{definition}

\begin{definition}[Ensemble ouvert]
    \label{def:ensemble_ouvert}
    Un ensemble est dit ouvert si tous ces éléments sont des \hyperref[def:point_int]{points intérieurs.}
\end{definition}

\begin{definition}[Groupe abélien]
    \label{def:groupe_abelien}
    Un \textbf{groupe abélien} ou \textbf{groupe non-commutatif} est un groupe
    (une structure algébrique associative avec un élément neutre et un élément inverse $(\forall a \in G, \exists a^{-1})$), 
    dont l'opération binaire est commutative.
\end{definition}

\begin{definition}[Famille d'ensemble]
    \label{def:famille}
    Une \textbf{famille} ou \textbf{collection} est simplement un ensemble d'ensemble.
\end{definition}

\begin{definition}[Compacité]
    \label{def:compacite}
    Soit $S$, un sous-ensemble d'une topologie. On dit que $S$ est \textbf{compact} si de tout recouvrement de $S$ par des ouverts, 
    on peut extraire un sous-recouvrement fini d'ouverts.
\end{definition}

\end{document}