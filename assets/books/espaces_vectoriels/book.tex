\documentclass[12pt]{report}

%\usepackage[french]{babel}
\usepackage[utf8]{inputenc} % entre autres, mettre les «...»
\usepackage[T1]{fontenc} % Output font encoding for international characters

\usepackage{amsmath}
\usepackage{amssymb}
\usepackage{amsfonts}
\usepackage{amscd} %diagramme commutatif
\usepackage{amsthm}
\usepackage{thmtools}
\usepackage[most]{tcolorbox}

\usepackage{latexsym}

\usepackage{fancybox} %boites ombrag\'{e}es
\usepackage{graphicx}
\usepackage[left=3cm,right=3cm,top=3cm,bottom=3cm]{geometry}
\usepackage{float}
\usepackage{alltt}
\usepackage{comment}
\usepackage{array}

\usepackage{graphicx}
\usepackage{multirow}
\usepackage{relsize}
\usepackage{hyperref}
\usepackage{xcolor}
\usepackage{xr-hyper} % Required for cross referencing
\usepackage{hyperref}

% EXTERNAL REFERENCES
% \externaldocument[topologie/]{../topologie/book}
% ----

\hypersetup{
    colorlinks,
    linkcolor={red!50!black},
    citecolor={blue!50!black},
    urlcolor={blue!80!black}
}
\newcommand{\overbar}[1]{\mkern 1.5mu\overline{\mkern-1.5mu#1\mkern-1.5mu}\mkern 1.5mu}

\newcommand*\conj[1]{\overbar{#1}}
% STATS Shortcut
\let\Bbb\mathbb
\def\sep{\phantom{}}
\newcommand\todo[1]{\phantom{#1}}
\theoremstyle{definition}
\newtheorem{definition}{Définition}[section]
\newtheorem*{example}{Exemple}
\newtheorem{theorem}{Theorême}[section]
\newtheorem{corollary}{Corollaire}[theorem]
\newtheorem{lemma}[theorem]{Lemme}


\declaretheoremstyle[
    headpunct={:\\[6pt]},
    %bodyfont=\slshape,
    ]{boxed}
\declaretheorem[style=boxed, name=Propriété, numberwithin=chapter]{prop}
\newenvironment{preuve}{\renewcommand{\proofname}{Preuve}\begin{proof}}{\end{proof}\noindent\textcolor[RGB]{220,220,220}{\rule{\textwidth}{1pt}}}


\tcolorboxenvironment{prop}{
    enhanced jigsaw,
    colback=white,
    drop shadow=black,
    sharp corners,
    coltitle=black,
    colbacktitle=red,
    before skip=12pt,
    after skip=24pt,
    left skip=12pt,
    right skip=12pt,
    bottom=12pt
}

\declaretheorem[style=boxed, name=Définition, numberwithin=chapter]{defi}
\tcolorboxenvironment{defi}{
    enhanced jigsaw,
    colback=white,
    drop shadow=black,
    sharp corners,
    coltitle=black,
    colbacktitle=red,
    before skip=12pt,
    after skip=24pt,
    left skip=12pt,
    right skip=12pt,
    bottom=12pt
}

\newtcolorbox{attention}{
    colback=black!5!white,
    coltitle={black},
    frame hidden,
    rounded corners,
    boxrule=0pt,
    boxsep=0pt,
    breakable,
    enlarge bottom by=0.3cm,
    enhanced jigsaw,
    borderline west={3pt}{0pt}{black!70!white},
    title={Attention! \linebreak},
    fonttitle={\small\bfseries}, 
    attach title to upper,
    before skip=12pt,
    after skip=12pt,
}

\renewcommand*{\chaptername}{Chapitre}

\title{Espaces Vectoriels}

\begin{document}
\chapter{Espace des fonctions Linéaires}
\section{La fonction linéaire}
    % [MOTIVATION]
    Une catégorie de fonctions importante est la catégorie des fonctions linéaires.
    Ces fonctions représentent intrinséquement le concept de proportionnalité. Dans 
    le monde des fonctions $\Bbb R$, ce sont les fonctions du type $y = ax + b$. Maintenant,
    nous allons nous attarder aux fonctions dans $\Bbb R^m$.
    % [DÉFINITIONS]
    \begin{defi}[Fonction linéaire]
        \label{def:fonction_lineaire}
        Une \textbf{fonction linéaire} ou \textbf{application linéaire} $L: \Bbb R^m \mapsto \Bbb R^n$ est une fonction possèdant
        ces deux propriétés:
        \begin{enumerate}
            \setlength{\itemsep}{0pt}
            \item Addition: $L(x + y) = L(x) + L(y)$
            \item Multiplication par un scalaire: $L(ax) = aL(x)$
        \end{enumerate}
    \end{defi}
    Afin de pouvoir mieux traiter des fonctions linéaires, il est nécessaire
    de définir le concept de matrice associée qui représente le coefficient de notre 
    fonction linéaire.
    \begin{defi}[Matrice associée]
        \label{def:matrice_associee}
    \end{defi}
    Démontrons maintenant que l'équation $L(x) = Cx$ est bel et bien une \hyperref[def:fonction_lineaire]{fonction linéaire}:
    \begin{preuve}
    \end{preuve}
    % [PROPRIÉTÉS]
    Commençons tout d'abord par définir deux propriétés des fonctions linéaires qui nous 
    permettrons d'établir que l'on peut "construire" des fonctions linéaires à partir 
    d'autres fonctions linéaire.
    \begin{prop}[Addition de fonctions linéaires]
        \label{thm:addition_fonction_lineaire}
        Soit $L_1(x)$ et $L_2(x)$, deux fonctions linéaires de $\Bbb R^m \mapsto \Bbb R^n$.
        l'addition de ces deux fonctions, $L_1(x) + L_2(x)$, est aussi une fonction linéaire.
    \end{prop}
    \begin{preuve}
    \end{preuve}
    
    \begin{prop}[Multiplication par un scalaire]
        \label{thm:multiplication_scalaire}
        Soit $L(x)$ une fonction linéaire de $\Bbb R^m \mapsto \Bbb R^n$  et $c \in \Bbb R$.
        la multiplication de cette fonction par le scalaire $c$, $c \cdot L(x)$, est aussi une fonction linéaire.
    \end{prop}
    \begin{preuve}
    \end{preuve}

    \subsection{Compositions des fonctions linéaires}
    Afin de continuer dans notre "construction" de fonctions linéaires, attardons-nous
    maintenant à la composition de celles-ci.
    \begin{prop}[Composition fonction linéaire]
        \label{thm:composition_fonction_lineaire}
        Soit $L_1(x)$, une fonction linéaire de $\Bbb R^m \mapsto \Bbb R^k$ et $L_2(x)$, une fonction linéaire de $\Bbb R^k \mapsto \Bbb R^n$.
        la composition de ces deux fonctions, $L_2(x) \circ L_1(x)$, est aussi une fonction linéaire ($\Bbb R^m \mapsto \Bbb R^n$).
    \end{prop}
    \begin{preuve}
    \end{preuve}
    Il en découle un résultat intéressant sur la matrice associée à cette composition.
    \begin{prop}[Composition des matrices associées]
        \label{thm:composition_matrices_associes}
        Soit $L_1(x)$, une fonction linéaire de $\Bbb R^m \mapsto \Bbb R^k$, $A$ sa matrice associée de taille $k \times m$,
         $L_2(x)$, une fonction linéaire de $\Bbb R^k \mapsto \Bbb R^n$ et $B$ sa matrice associée de taille $n \times k$.
         \\
         La matrice associée à la composition $L_2(x) \circ L_1(x)$ sera la matrice $C = BA$
    \end{prop}
    \begin{preuve}
    \end{preuve}

    \subsection{Normes et longueurs de fonctions linéaires}
    N'oublions pas que puisque l'image d'une fonction linéaire est un 
    espace vectoriel, il est possible d'établir sa norme. Ainsi on peut découler
    deux propriétés intéressantes
    \begin{prop}[Principe archimédien]
        \label{thm:principe_archimedien}
        Soit $L$ une fonction linéaire, il existe une constante $M$ tq:
        $$ ||L(x)|| \leq M \cdot ||x||, \forall x $$
    \end{prop}
    \begin{preuve}
    \end{preuve}
    On peut alors découler directement une forme du principe archimédien qui 
    nous sera plus utile.
    \begin{prop}
        \label{thm:principe_archimedien_2}
        Soit $L$ une fonction linéaire de domaine $\Bbb R^n$ et $u, v$ des vecteurs de $\Bbb R^n$, il existe une constante $M$ tq:
        $$ ||L(u) - L(v)|| \leq M \cdot ||u - v||, \forall u,v $$
    \end{prop}
    \subsection{Continuité des fonctions linéaires}
    Suite aux résultats précédents, on peut maintenant affirmer que:
    \begin{prop}[Continuité uniforme des fonctions linéaires]
        \label{thm:continuite_uniforme_fonction_lineaire}
        Une fonction linéaire est uniformément continue.
    \end{prop}
    \begin{preuve}
    \end{preuve}
\section{L'ensemble des fonctions linéaires}
    % [MOTIVATION]
    Lorsque l'on parle plutôt de l'espace des fonctions linéaires, on utilisera plutôt
    la notation suivante:

    \begin{defi}[Ensemble des fonctions linéaires]
        \label{def:ensemble_fonctions_lineaires}
        L'\textbf{ensemble des fonctions linéaires} de $\Bbb R^m$ vers $\Bbb R^n$ se note:
        $$\mathcal{L}(\Bbb R^m, \Bbb R^n)$$
    \end{defi}

    % [PROPRIÉTÉS]
    \begin{prop}
        \label{thm:ensemble_fonctions_lineaires_espace_Vectoriels}
        L'\hyperref[def:ensemble_fonctions_lineaires]{ensemble des fonctions linéaires} forme un espace vectoriel \todo{Add link here}
    \end{prop}
    \begin{preuve}
        Contrary to popular belief, Lorem Ipsum is not simply random text. 
        It has roots in a piece of classical Latin literature from 45 BC, 
        making it over 2000 years old. Richard McClintock, a Latin professor at Hampden-Sydney College i
    \end{preuve}

    \begin{attention}
        To be or not to be, that is the question!
    \end{attention}

    % [APPLICATIONS ET EXEMPLES]

    \begin{example}
        Ceci est un exemple
    \end{example}
\chapter{La dérivée}
\section{La dérivée dans $\Bbb R^n$}
    % [MOTIVATION]
    Il es plus difficile d'imaginer ce que représente une dérivée dans un espace vectoriel 
    quelconque. En effet, nous ne parlons plus d'une simple pente de courbe. La dérivée
    peut venir de multiple côtés. Essayons-donc de définir la notion de dérivée dans $\Bbb R^n$
    en se basant sur la définition dans $\Bbb R$.
    % [DÉFINITIONS]
    \begin{defi}[Dérivée de $\Bbb R^n \mapsto \Bbb R^n$]
        \label{def:derivee_rn}

        La \textbf{dérivée} $L$ au point intérieur $c$ est un vecteur tq 
        $$ \forall \epsilon > 0, \exists \delta > 0, \text{ tq } 0 < ||x - c|| < \delta \implies  \big| \big| \frac{f(x) - f(c) - L\cdot(x - c)}{x - c} \big|\big|< \epsilon $$
    \end{defi}
    \begin{attention}
        Cette définition reste valide et applicable pour la dérivé dans les $\Bbb R$.
    \end{attention}

    Afin de généraliser la définition dans les $\Bbb R^n \mapsto \Bbb R^n$, ce que nous avons fait est plutôt simple:
    \todo{add link}
    \begin{enumerate}
        \setlength{\itemsep}{0pt}
        \item Nous avons utiliser la norme, puisque la valeur absolute n'a pas de sens sur un vecteur.
        \item Nous avons "enmener" le $L$ au dénominateur en le multipliant par le vecteur $x - c$
    \end{enumerate}

    Il nous reste maintenant à généraliser cette définition pour les fonctions $\Bbb R^m \mapsto \Bbb R^n$. Pour ce faire, il est crucial de remarquer que 
    le produit $L \cdot (x - c)$ peut être perçu comme une \hyperref[def:fonction_lineaire]{application linéaire} sur $x - c$. ($L$ agissant comme une "constante" qu'on multiplie)
    \begin{defi}[Dérivée de $\Bbb R^n \mapsto \Bbb R^m$]
        \label{def:derivee_generalise}
        La \textbf{dérivée} $L$ au point intérieur $c$ est un \hyperref[def:fonction_lineaire]{application linéaire} tq 
        $$ \forall \epsilon > 0, \exists \delta > 0, \text{ tq } 0 < ||x - c|| < \delta \implies \frac{||f(x) - f(c) - L(x - c)||}{||x - c||} < \epsilon $$
    \end{defi}
    Ici, afin de bien généraliser pour l'ensemble des fonctions dans les espaces vectoriels, nous avons fait deux changements additionnels:
    \begin{enumerate}
        \setlength{\itemsep}{0pt}
        \item Nous avons "distribué" la norme en haut et en bas, car $f$ étant de $\Bbb R^n \mapsto \Bbb R^m$, $x - c$ est un vecteur de $\Bbb R^n$ et $f(x) - f(c)$ un vecteur de $\Bbb R^m$.
        \item Transformer le produit $L \cdot (x - c)$ en application linéaire ce qui nous permet d'avoir un vecteur de $m$ dimensions.
    \end{enumerate}

    Finalement, il est pratique d'avoir une définition avec limite, qui est souvent plus pratique à travailler:
    \begin{defi}[Dérivée de $\Bbb R^n \mapsto \Bbb R^m$ avec limites]
        \label{def:derivee_limite}
        La \textbf{dérivée} $L$ au point intérieur $c$ est un \hyperref[def:fonction_lineaire]{application linéaire} tq 
        $$ \lim_{h \to 0} \frac{||f(c + h) - f(c) - L(h)||}{||h||} = 0$$
    \end{defi}
    \begin{attention}
        Nous avons obtenu notre $h$ (un vecteur) en soustrayant $x - c$.
    \end{attention}

    Il est souvent plus utile de parler en terme d'une fonction qui nous retourne 
    cette application linéaire. Tout d'abord telle que vu précédemment, une 
    fonction linéaire peut être caractérisée par une matrice $m \times n$.
    \\
    On notera donc $L$ pour $f$ comme suit:
    $$ L = Df(c) $$
    où $c$ est un point intérieur du domaine de $f$ et $L$ est une matrice $m \times n$.\\
    \pagebreak

    % [PROPRIÉTÉS]
    Attardons-nous maintenant aux propriétés de cette dérivée dans l'espace vectorielle.
    \begin{prop}[La dérivée est unique]
        \label{thm:unicite_derivee}
        Soit $f: U \subseteq R^m \mapsto R^n, c \in int{U}$, la dérivée $Df(c)$, si elle existe,
        est \textbf{unique}.
    \end{prop}
    \begin{preuve}
    \end{preuve}

    \begin{prop}[La dérivée de la fonction constante]
        \label{thm:derivee_constante}
        Soit $f$ une fonction constante, la dérivée $Df(c)$ est la fonction
        linéaire zéro.
    \end{prop}
    \begin{preuve}
        Posons $f(x) = k, \forall x$. \\
        Ainsi $||f(x) - f(c) - L(x - c)|| = ||k - k - L(x - c)|| = ||L(x - c)||$. \\
        Si $(\forall \epsilon > 0), ||L(x - c)|| < \epsilon \implies L(x - c) = 0$.
    \end{preuve}

    \begin{prop}[La dérivée de la fonction linéaire]
        \label{thm:derivee_constante}
        Soit $f$ une fonction linéaire, la dérivée $Df(c)$ est la fonction
        linéaire elle même.
    \end{prop}
    \begin{preuve}
        $||f(x) - f(c) - L(x - c)|| = ||f(x - c) - L(x - c)|| = ||L(x - c)||$. \\
        Si $(\forall \epsilon > 0), ||f(x - c) - L(x - c)|| < \epsilon \implies L = f$.
    \end{preuve}
    % [APPLICATIONS ET EXEMPLES]

    \begin{example}
        Soit $f: \Bbb R^2 \mapsto \Bbb R^3$ tel que $f(x,y) = (xy + 1, x + y, x^2 + 1)$.
        Démontrons que la dérivée $f$ à $c = (1,2)$ est la fonction linéaire: $L(u,v) = (2u + v, u + v, 2u)$
        \begin{align*}
            &||f(x) - f(c) - L(x - c)|| \\
            &= ||(f(x,y) - f(1,2) - L(x - 1, y - 2)) ||\\
            &= || (xy + 1, x + y, x^2 + 1) - (1\cdot 2 + 1, 1 + 2, 1^2 + 1) - (2(x - 1) + (y - 2), (x - 1) + (y - 2), 2(x - 1)) ||\\
            &= || (xy + 1 - 3 - 2(x - 1) - (y - 2), x + y - 3 - (x - 1) - (y - 2), x^2 + 1 - 2 - 2(x - 1)) ||\\
            &= || (xy + 2 -2x -y, 0, x^2 - 2x + 1)|| \\
            &= || ((x - 1)(y - 2), 0, (x - 1)^2)|| \\
            &= \sqrt{(x - 1)^2(y - 2)^2 + 0^2 + (x - 1)^4} \\
            &= \sqrt{(x - 1)^2 \cdot ((y - 2)^2 + (x - 1)^2)} \\
            &= |(x - 1)| \cdot \sqrt{(y - 2)^2 - (x - 1)^2}
        \end{align*}
        Quant à elle la norme de $|| x - c ||$:
        \begin{align*}
            &||x - c|| \\
            &= || (x, y) - (1, 2) || \\
            &= ||(x - 1, y - 2)|| \\
            &= \sqrt{(x - 1)^2 + (y - 2)^2}
        \end{align*}
        Ainsi:
        \begin{align*}
            &\frac{||f(x) - f(c) - L(x - c)||}{||x - c||}\\
            &=\frac{|(x - 1)| \cdot \sqrt{(y - 2)^2 - (x - 1)^2}}{\sqrt{(x - 1)^2 + (y - 2)^2}}\\
            &= |(x - 1)|\\
            &= \sqrt{(x - 1)^2}
        \end{align*}
        En prenant un $\delta = \epsilon$ et puisque $\sqrt{(x - 1)^2} \leq \sqrt{(x - 1)^2 + (y - 2)^2}$, on obtient:
        $$ || x - c || < \delta \implies \sqrt{(x - 1)^2} \leq \sqrt{(x - 1)^2 + (y - 2)^2} < \delta \implies \sqrt{(x - 1)^2} < \epsilon $$
    \end{example}
    
\chapter{La dérivée partielle}
\section{La dérivée partielle}
    % [MOTIVATION]
    Comme observé précédemment, il est difficile de travailler directement avec la dérivé elle-même
    dans une fonction $R^m$. Il sera donc plus facile de traiter avec un objet mathématique plus spécifique:
    la dérivée partielle.    
    % [DÉFINITIONS]
    \begin{defi}[Dérivée partielle]
        \label{def:derivee_partielle_1}
        Soit $f: A \subseteq \Bbb R^m \mapsto \Bbb R$, $A$ ouvert et $c$ un vecteur $\in R^m$.
        La \textbf{dérivée partielle} de $f$, $\frac{\partial f}{\partial x_i}(c)$ ou $D_i f(c)$
        est la fonction telle que:
        $$\frac{\partial f}{\partial x_i}(c) = \lim_{h_i \to 0} \frac{f(c + (0,\dots,h_i,\dots,0)) - f(c)}{h_i}$$
    \end{defi}
    Alternativement, on peut définir la dérivée partielle de cette façon:
    \begin{defi}[Dérivée partielle]
        \label{def:derivee_partielle_1}
        Soit $f: A \subseteq \Bbb R^m \mapsto \Bbb R$, $A$ ouvert et $c$ un vecteur $\in R^m$.
        La \textbf{dérivée partielle} de $f$, $\frac{\partial f}{\partial x_i}(c)$ ou $D_i f(c)$
        est la fonction telle que:
        $$\frac{\partial f}{\partial x_i}(c) = \lim_{h_i \to 0} \frac{f(c + (0,\dots,h_i,\dots,0)) - f(c)}{h_i}$$
    \end{defi}

    Or alternatively, a list:
    \begin{enumerate}
        \setlength{\itemsep}{0pt}
        \item Item 1
        \item Item 2
    \end{enumerate}

    % [PROPRIÉTÉS]
    \begin{prop}[THEOREM NAME OPTIONAL]
        \label{thm:THEOREM}
        Lorem Ipsum
    \end{prop}
    \begin{preuve}
        Contrary to popular belief, Lorem Ipsum is not simply random text. 
        It has roots in a piece of classical Latin literature from 45 BC, 
        making it over 2000 years old. Richard McClintock, a Latin professor at Hampden-Sydney College i
    \end{preuve}

    \begin{attention}
        To be or not to be, that is the question!
    \end{attention}

    % [APPLICATIONS ET EXEMPLES]

    \begin{example}
        Ceci est un exemple
    \end{example}

\appendix
\chapter{Notions préalables}
    Voici une série de définition importantes à considérer lors de la lecture de cet ouvrage. Celles-ci servent
    de références pour les exemples et les preuves qui seront amenés dans les chapitres suivants. Pour une explication
    plus complète et rigoureuse, il sera nécessaire de se réferrer aux livre précédent: Analyse des Réels.

    \begin{definition}[Dérivé dans les réels]
        \label{def:derive_reels}
        La \textbf{dérivée} $L$ au point intérieur $c$ est la valeur tq 
        $$ \forall \epsilon > 0, \exists \delta > 0, \text{ tq } 0 < |x - c| < \delta \implies \big| \frac{f(x) - f(c)}{x - c} - L \big| < \epsilon $$
    \end{definition}

\end{document}