\documentclass[12pt]{report}

%\usepackage[french]{babel}
\usepackage[utf8]{inputenc} % entre autres, mettre les «...»
\usepackage[T1]{fontenc} % Output font encoding for international characters

\usepackage{amsmath}
\usepackage{amssymb}
\usepackage{amsfonts}
\usepackage{amscd} %diagramme commutatif
\usepackage{amsthm}
\usepackage{thmtools}
\usepackage[most]{tcolorbox}

\usepackage{latexsym}

\usepackage{fancybox} %boites ombrag\'{e}es
\usepackage{graphicx}
\usepackage[left=3cm,right=3cm,top=3cm,bottom=3cm]{geometry}
\usepackage{float}
\usepackage{alltt}
\usepackage{comment}
\usepackage{array}

\usepackage{graphicx}
\usepackage{multirow}
\usepackage{relsize}
\usepackage{hyperref}
\usepackage{xcolor}
\usepackage{xr-hyper} % Required for cross referencing
\usepackage{hyperref}

% EXTERNAL REFERENCES
% \externaldocument[topologie/]{../topologie/book}
% ----

\hypersetup{
    colorlinks,
    linkcolor={red!50!black},
    citecolor={blue!50!black},
    urlcolor={blue!80!black}
}
\newcommand{\overbar}[1]{\mkern 1.5mu\overline{\mkern-1.5mu#1\mkern-1.5mu}\mkern 1.5mu}

\newcommand*\conj[1]{\overbar{#1}}
% STATS Shortcut
\let\Bbb\mathbb
\def\sep{\phantom{}}
\newcommand\todo[1]{\phantom{#1}}
\theoremstyle{definition}
\newtheorem{definition}{Définition}[section]
\newtheorem*{example}{Exemple}
\newtheorem{theorem}{Theorême}[section]
\newtheorem{corollary}{Corollaire}[theorem]
\newtheorem{lemma}[theorem]{Lemme}  

\declaretheoremstyle[
    headpunct={:\\[6pt]},
    %bodyfont=\slshape,
    ]{boxed}
\declaretheorem[style=boxed, name=Propriété,]{prop}

\tcolorboxenvironment{prop}{
    enhanced jigsaw,
    colback=white,
    drop shadow=black,
    sharp corners,
    coltitle=black,
    colbacktitle=red,
    before skip=12pt,
    after skip=24pt,
    left skip=12pt,
    right skip=12pt,
    bottom=12pt
}

\declaretheorem[style=boxed, name=Définition]{defi}
\tcolorboxenvironment{defi}{
    enhanced jigsaw,
    colback=white,
    drop shadow=black,
    sharp corners,
    coltitle=black,
    colbacktitle=red,
    before skip=12pt,
    after skip=24pt,
    left skip=12pt,
    right skip=12pt,
    bottom=12pt
}

\newtcolorbox{attention}{
    colback=black!5!white,
    coltitle={black},
    frame hidden,
    rounded corners,
    boxrule=0pt,
    boxsep=0pt,
    breakable,
    enlarge bottom by=0.3cm,
    enhanced jigsaw,
    borderline west={3pt}{0pt}{black!70!white},
    title={Attention! \linebreak},
    fonttitle={\small\bfseries}, 
    attach title to upper,
    before skip=12pt,
    after skip=12pt,
}

\renewcommand*{\proofname}{Preuve}

\title{Espaces Vectoriels}

\begin{document}
\chapter{Espace des fonctions Linéaires}
\section{La fonction linéaire}
    % [MOTIVATION]
    Une catégorie de fonctions importante est la catégorie des fonctions linéaires.
    Ces fonctions représentent intrinséquement le concept de proportionnalité. Dans 
    le monde des fonctions $\Bbb R$, ce sont les fonctions du type $y = ax + b$. Maintenant,
    nous allons nous attarder aux fonctions dans $\Bbb R^m$.
    % [DÉFINITIONS]
    \begin{defi}[Fonction linéaire]
        \label{def:fonction_lineaire}
        Une \textbf{fonction linéaire} $L: \Bbb R^m \mapsto \Bbb R^n$ est une fonction possèdant
        ces deux propriétés:
        \begin{enumerate}
            \setlength{\itemsep}{0pt}
            \item Addition: $L(x + y) = L(x) + L(y)$
            \item Multiplication par un scalaire: $L(ax) = aL(x)$
        \end{enumerate}
    \end{defi}
    Afin de pouvoir mieux traiter des fonctions linéaires, il est nécessaire
    de définir le concept de matrice associée qui représente le coefficient de notre 
    fonction linéaire.
    \begin{defi}[Matrice associée]
        \label{def:matrice_associee}
    \end{defi}
    Démontrons maintenant que l'équation $L(x) = Cx$ est bel et bien une \hyperref[def:fonction_lineaire]{fonction linéaire}:
    \begin{proof}
    \end{proof}
    % [PROPRIÉTÉS]
    Commençons tout d'abord par définir deux propriétés des fonctions linéaires qui nous 
    permettrons d'établir que l'on peut "construire" des fonctions linéaires à partir 
    d'autres fonctions linéaire.
    \begin{prop}[Addition de fonctions linéaires]
        \label{thm:addition_fonction_lineaire}
        Soit $L_1(x)$ et $L_2(x)$, deux fonctions linéaires de $\Bbb R^m \mapsto \Bbb R^n$.
        l'addition de ces deux fonctions, $L_1(x) + L_2(x)$, est aussi une fonction linéaire.
    \end{prop}
    \begin{proof}
    \end{proof}
    
    \begin{prop}[Multiplication par un scalaire]
        \label{thm:multiplication_scalaire}
        Soit $L(x)$ une fonction linéaire de $\Bbb R^m \mapsto \Bbb R^n$  et $c \in \Bbb R$.
        la multiplication de cette fonction par le scalaire $c$, $c \cdot L(x)$, est aussi une fonction linéaire.
    \end{prop}
    \begin{proof}
    \end{proof}

    \subsection{Compositions des fonctions linéaires}
    Afin de continuer dans notre "construction" de fonctions linéaires, attardons-nous
    maintenant à la composition de celles-ci.
    \begin{prop}[Composition fonction linéaire]
        \label{thm:composition_fonction_lineaire}
        Soit $L_1(x)$, une fonction linéaire de $\Bbb R^m \mapsto \Bbb R^k$ et $L_2(x)$, une fonction linéaire de $\Bbb R^k \mapsto \Bbb R^n$.
        la composition de ces deux fonctions, $L_2(x) \circ L_1(x)$, est aussi une fonction linéaire ($\Bbb R^m \mapsto \Bbb R^n$).
    \end{prop}
    \begin{proof}
    \end{proof}
    Il en découle un résultat intéressant sur la matrice associée à cette composition.
    \begin{prop}[Composition des matrices associées]
        \label{thm:composition_matrices_associes}
        Soit $L_1(x)$, une fonction linéaire de $\Bbb R^m \mapsto \Bbb R^k$, $A$ sa matrice associée de taille $k \times m$,
         $L_2(x)$, une fonction linéaire de $\Bbb R^k \mapsto \Bbb R^n$ et $B$ sa matrice associée de taille $n \times k$.
         \\
         La matrice associée à la composition $L_2(x) \circ L_1(x)$ sera la matrice $C = BA$
    \end{prop}
    \begin{proof}
    \end{proof}

    \subsection{Normes et longueurs de fonctions linéaires}
    N'oublions pas que puisque l'image d'une fonction linéaire est un 
    espace vectoriel, il est possible d'établir sa norme. Ainsi on peut découler
    deux propriétés intéressantes
    \begin{prop}[Principe archimédien]
        \label{thm:principe_archimedien}
        Soit $L$ une fonction linéaire, il existe une constante $M$ tq:
        $$ ||L(x)|| \leq M \cdot ||x||, \forall x $$
    \end{prop}
    \begin{proof}
    \end{proof}
    On peut alors découler directement une forme du principe archimédien qui 
    nous sera plus utile.
    \begin{prop}
        \label{thm:principe_archimedien_2}
        Soit $L$ une fonction linéaire de domaine $\Bbb R^n$ et $u, v$ des vecteurs de $\Bbb R^n$, il existe une constante $M$ tq:
        $$ ||L(u) - L(v)|| \leq M \cdot ||u - v||, \forall u,v $$
    \end{prop}
    \subsection{Continuité des fonctions linéaires}
    Suite aux résultats précédents, on peut maintenant affirmer que:
    \begin{prop}[Continuité uniforme des fonctions linéaires]
        \label{thm:continuite_uniforme_fonction_lineaire}
        Une fonction linéaire est uniformément continue.
    \end{prop}
    \begin{proof}
    \end{proof}
\section{L'ensemble des fonctions linéaires}
    % [MOTIVATION]
    Lorsque l'on parle plutôt de l'espace des fonctions linéaires, on utilisera plutôt
    la notation suivante:

    \begin{defi}[Ensemble des fonctions linéaires]
        \label{def:ensemble_fonctions_lineaires}
        L'\textbf{ensemble des fonctions linéaires} de $\Bbb R^m$ vers $\Bbb R^n$ se note:
        $$\mathcal{L}(\Bbb R^m, \Bbb R^n)$$
    \end{defi}

    % [PROPRIÉTÉS]
    \begin{prop}
        \label{thm:ensemble_fonctions_lineaires_espace_Vectoriels}
        L'\hyperref[def:ensemble_fonctions_lineaires]{ensemble des fonctions linéaires} forme un espace vectoriel \todo{Add link here}
    \end{prop}
    \begin{proof}
        Contrary to popular belief, Lorem Ipsum is not simply random text. 
        It has roots in a piece of classical Latin literature from 45 BC, 
        making it over 2000 years old. Richard McClintock, a Latin professor at Hampden-Sydney College i
    \end{proof}

    \begin{attention}
        To be or not to be, that is the question!
    \end{attention}

    % [APPLICATIONS ET EXEMPLES]

    \begin{example}
        Ceci est un exemple
    \end{example}

\end{document}