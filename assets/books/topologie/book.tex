\documentclass[12pt]{book}

\usepackage[utf8]{inputenc} % Required for inputting international characters
\usepackage[T1]{fontenc} % Output font encoding for international characters

\usepackage{comment}
\usepackage{mathpazo} % Palatino font
\usepackage{amssymb}
\usepackage{amsmath}
\usepackage{array}
\usepackage{amsthm}
\usepackage{graphicx}
\usepackage{multirow}
\usepackage{relsize}
\usepackage{hyperref}
\usepackage{xcolor}
\hypersetup{
    colorlinks,
    linkcolor={red!50!black},
    citecolor={blue!50!black},
    urlcolor={blue!80!black}
}
\usepackage[makeroom]{cancel}

\usepackage[margin=0.5in]{geometry}
\newcommand{\overbar}[1]{\mkern 1.5mu\overline{\mkern-1.5mu#1\mkern-1.5mu}\mkern 1.5mu}

\newcommand*\conj[1]{\overbar{#1}}
% STATS Shortcut
\newcommand{\BP}{\Bbb P}
\newcommand{\BR}{\Bbb R}
\let\Bbb\mathbb
\def\sep{\phantom{}}
\newcommand\todo[1]{\phantom{#1}}
\theoremstyle{definition}
\newtheorem{definition}{Définition}[section]
\newtheorem*{example}{Exemple}
\newtheorem{theorem}{Theorême}[section]
\newtheorem{corollary}{Corollaire}[theorem]
\newtheorem{lemma}[theorem]{Lemme}  

\title{Topologie}

\begin{document}
\chapter{Introduction}
Qu'est-ce que la topologie?


\chapter{Espaces vectoriels}
\todo{Complete}
\begin{definition}
    \label{def:espace_vectoriel} 
    Soit $K$ un corps commutatif, Un \textbf{espace vectoriel} est un ensemble de vecteurs $V$ muni de deux opérations binaires
\end{definition}

\section{Addition de vecteurs}
\begin{itemize}
    \item 
\end{itemize}

\section{Multiplication par un scalaire}
\begin{itemize}
    \item 
\end{itemize}

\section{Produit scalaire}
\begin{definition}
    \label{def:produit_scalaire}
\end{definition}
\subsection{Produit scalaire dans R}
\begin{definition}
    \label{def:produit_scalaire_r}
\end{definition}
\subsection{Produit scalaire dans Rn}
\begin{definition}
    \label{def:produit_scalaire_rn}
\end{definition}


\chapter{Inégalité de Cauchy-Schwarz}
\begin{lemma}[Inégalité de Cauchy-Schwarz]
    Soit $a_1,a_2,\dots,a_n$ et $b_1,b_2,\dots,b_n$ des nombre réels arbitraires, alors:
    $$ (\sum_{i=1}^{n}a_ib_i)^2 \leq (\sum_{i=1}^{n}a_i^2)(\sum_{i=1}^{n}b_i^2) $$   
\end{lemma}

\begin{proof}
\end{proof}


\chapter{Espaces normés}

\section{Norme}
\begin{definition}
    \label{def:norme}
\end{definition}

\subsection{Norme dans R}

\subsection{Norme dans Rn}

\subsection{Norme par le produit scalaire}

\section{Espace préhilbertien}

\chapter{Espaces métriques}
\section{Les métriques}
\begin{definition}
    \label{def:metrique}
\end{definition}

\section{Convergence des suites}
\section{Bornes}
\section{Suites de Cauchy}
\section{Complétude}
\section{Continuité}
\section{Contractions}
\section{Points fixes}


\chapter{Espaces topologiques}
\section{Topologie}
\section{Boule}
\section{Voisinage}
\section{Point intérieur}
\section{Ouverture}


\appendix
\chapter{Notions préalables}
Voici une série de définition importantes à considérer lors de la lecture de cet ouvrage. Celles-ci servent
de références pour les exemples et les preuves qui seront amenés dans les chapitres suivants. Pour une explication
plus complète et rigoureuse, il sera nécessaire de se réferrer aux livre précédent: Analyse des Réels.

\begin{definition}[Voisinage dans les Réels]
    \label{def:voisinage_reels}
    Ensemble de point satisfaisant l'expression
    suivante: $$V(a, \delta) = \{ x \in \Bbb R : |x - a| < \delta \}$$
\end{definition}

\begin{definition}[Voisinage troué dans les Réels]
    \label{def:voisinage_troue_reels}
    Ensemble de point satisfaisant l'expression
    suivante: $$V'(a, \delta) = \{ x \in \Bbb R : |x - a| < \delta \land x \neq a \}$$
\end{definition}

\begin{definition}[Points intérieurs de E]
    \label{def:point_int}
    Ensemble de points satisfaisant l'expression
    suivante: $$int(E) = \{ x \in E : \exists \delta > 0, V(x, \delta) \subseteq E \} $$
\end{definition}

\begin{definition}[Ensemble ouvert]
    \label{def:ensemble_ouvert}
    Un ensemble est dit ouvert si tous ces éléments sont des \hyperref[def:point_int]{points intérieurs.}
\end{definition}

\begin{definition}[Groupe abélien]
    \label{def:groupe_abelien}
    Un \textbf{groupe abélien} ou \textbf{groupe non-commutatif} est un groupe
    (une structure algébrique associative avec un élément neutre et un élément inverse $(\forall a \in G, \exists a^{-1})$), 
    dont l'opération binaire est commutative.
\end{definition}
\todo{Add reference}

\chapter{Cheatsheet}
\section{Formules}
\subsection{Inégalité de Cauchy-Schwarz}
$$(\sum_{i=1}^{n} a_ib_i)^2 \leq \sum_{i=1}^{n} a_i^2 \cdot \sum_{i=1}^{n} b_i ^ 2$$ 
\subsection{Convergence série géométrique}
$$|r| < 1 \implies \sum_{n=0}^{\infty} r^n \to \frac{1}{1 - r} $$
\subsection{Inégalité du triangle}
$$| ||x|| - ||y|| | \leq ||x \pm y|| \leq ||x|| + ||y|| $$
\section{Types d'espace}
\begin{itemize}
    \item Espace vectoriel: Ensemble d'éléments appellés vecteurs, muni de l'addition et de la multiplication par un scalaire.
    \item Espace préhilbertien: Espace vectoriel muni d'un produit scalaire.
    \item Espace normé: Espace vectoriel muni d'une norme.
    \item Espace métrique: Ensemble \textbf{quelconque} munie d'une métrique.
    \item Espace métrique complet:
    \item Espace topologique:
\end{itemize}
\section{Fonctions spéciales}
\subsection{Produit scalaire}
\begin{itemize}
    \item Un produit scalaire est une fonction de $V \times V \to \Bbb R$, noté $\langle x, y \rangle$ avec 5 propriétés qui la définisse
    \item $\langle x, x \rangle \geq 0$
    \item $\langle x, x \rangle = 0 \iff x = 0$
    \item $\langle x, y \rangle = \langle y, x \rangle$
    \item $\langle x + y, z\rangle = \langle x, z \rangle + \langle y, z \rangle$
    \item $\alpha\langle x, y \rangle = \langle \alpha x, y \rangle \forall \alpha \in \Bbb R$
\end{itemize}
\subsection{Norme}
\begin{itemize}
    \item Une norme sur $V$ est une fonction $V$ vers $\Bbb R$, noté $x \mapsto ||x||$ avec 4 propriétés qui la définisse:
    \item $||x|| \geq 0$
    \item $||x|| = 0 \iff x = 0$
    \item $||\alpha x|| = |\alpha| \cdot ||x||$
    \item $||x + y || \leq ||x|| + ||y||$
    \item Norme euclidienne: $\sqrt{x_1^2 + x_2^2 + \dots + x_n^2}$
    \item Le produit scalaire est une norme.
\end{itemize}
\subsection{Métrique}
\begin{itemize}
    \item Une métrique est une fonction $X \times X \to [0, \infty$ avec les 4 propriétés suivantes:
    \item $d(x,y) \geq 0$
    \item $d(x,y) = d(y, x)$
    \item $d(x,y) = 0 \iff x = y$
    \item $d(x,y) \leq d(x, z) + d(z, y)$
    \item Métrique discrète: $d(x,y) = 0$ si $x = y$, 1 si $x \neq 1$
    \item La norme est une métrique
\end{itemize}
\section{Définition topologiques}
\begin{itemize}
    \item \textbf{Boule ouverte} : $B(x, r) = \{y \in R^n : || x - y || < r\}, r > 0$
    \item \textbf{Boule fermée} : $B(x, r) = \{y \in R^n : || x - y || \leq r\}, r > 0$
    \item \textbf{Point intérieur}: $x \in \Bbb R^n \text{ est un point intérieur de } E \iff \exists \delta > 0 : B(x, \delta) \subset E$
    \item \textbf{Ensemble ouvert}: $E\text{ est ouvert} \iff E = int(E)$
    \item \textbf{Convergence suite}: $\forall \epsilon > 0, \exists M \in \Bbb N \text{ t q} m \geq M \implies d(x_m, x) < \epsilon$
    \item \textbf{Ensemble borné}: $\exists x_0 \in X \land \exists r > 0, \forall x \in A, d(x_0, x) < r$
\end{itemize}

\chapter{Exercices}
\section{Examen 1}
\subsection{Cours 2}
2.1 Soit $V$, un espace vectoriel réel sur lequel on a défini un produit scalaire et posons $||x|| = \sqrt{\langle x, x \rangle}$
2.1 a) Montrer que $\langle x, y \rangle \leq ||x|| \cdot ||y||$
\begin{align*}
    & \langle x, y \rangle \leq \langle ||x||x, ||y||y \rangle  \\
    & \langle x, y \rangle \leq ||x||||y||\langle x, y \rangle \\
    \implies & \langle x, y \rangle \leq \sqrt{\langle x, x \rangle} \cdot \sqrt{\langle y, y \rangle} \\
    \implies & \langle x, y \rangle^2 \leq \langle x, x \rangle \cdot \langle y, y \rangle \\
\end{align*}

\end{document}