\documentclass[12pt]{book}

\usepackage[utf8]{inputenc} % Required for inputting international characters
\usepackage[T1]{fontenc} % Output font encoding for international characters

\usepackage{comment}
\usepackage{mathpazo} % Palatino font
\usepackage{amssymb}
\usepackage{amsmath}
\usepackage{array}
\usepackage{amsthm}
\usepackage{graphicx}
\usepackage{multirow}
\usepackage{relsize}
\usepackage{hyperref}
\usepackage{xcolor}
\hypersetup{
    colorlinks,
    linkcolor={red!50!black},
    citecolor={blue!50!black},
    urlcolor={blue!80!black}
}
\usepackage[makeroom]{cancel}

\usepackage[margin=0.5in]{geometry}
\newcommand{\overbar}[1]{\mkern 1.5mu\overline{\mkern-1.5mu#1\mkern-1.5mu}\mkern 1.5mu}

\newcommand*\conj[1]{\overbar{#1}}
% STATS Shortcut
\newcommand{\BP}{\Bbb P}
\newcommand{\BR}{\Bbb R}
\let\Bbb\mathbb
\def\sep{\phantom{}}
\theoremstyle{definition}
\newtheorem{definition}{Définition}[section]
\newtheorem*{example}{Exemple}
\newtheorem{theorem}{Theorême}[section]
\newtheorem{corollary}{Corollaire}[theorem]
\newtheorem{lemma}[theorem]{Lemme}

\title{Topologie}

\begin{document}
\chapter{Introduction}
Qu'est-ce que la topologie?
\chapter{Notions préalables}
Voici une série de définition importantes à considérer lors de la lecture de cet ouvrage. Celles-ci servent
de références pour les exemples et les preuves qui seront amenés dans les chapitres suivants. Pour une explication
plus complète et rigoureuse, il sera nécessaire de se réferrer aux livre précédent: Analyse des Réels.

\begin{definition}[Voisinage dans $\Bbb R$]
    \label{def:voisinage_reels}
    Ensemble de point satisfaisant l'expression
    suivante: $$V(a, \delta) = \{ x \in \Bbb R : |x - a| < \delta \}$$
\end{definition}

\begin{definition}[Voisinage troué dans $\Bbb R$ os]
    \label{def:voisinage_troue_reels}
    Ensemble de point satisfaisant l'expression
    suivante: $$V'(a, \delta) = \{ x \in \Bbb R : |x - a| < \delta \land x \neq a \}$$
\end{definition}

\begin{definition}[Points intérieurs de E]
    \label{def:point_int}
    Ensemble de points satisfaisant l'expression
    suivante: $$int(E) = \{ x \in E : \exists \delta > 0, V(x, \delta) \subseteq E \} $$
\end{definition}
ts

\begin{definition}[Ensemble ouvert]
    \label{def:ensemble_ouvert}
    Un ensemble est dit ouvert si tous ces éléments sont des \hyperref[def:point_int]{points intérieurs.}
\end{definition}

Ceci est un \autoref{def:ensemble_ouvert}
\end{document}