\documentclass[12pt]{book}

\usepackage[utf8]{inputenc} % Required for inputting international characters
\usepackage[T1]{fontenc} % Output font encoding for international characters

\usepackage{comment}
\usepackage{mathpazo} % Palatino font
\usepackage{amssymb}
\usepackage{amsmath}
\usepackage{array}
\usepackage{amsthm}
\usepackage{graphicx}
\usepackage{multirow}
\usepackage{relsize}
\usepackage{hyperref}
\usepackage{xcolor}
\hypersetup{
    colorlinks,
    linkcolor={red!50!black},
    citecolor={blue!50!black},
    urlcolor={blue!80!black}
}
\usepackage[makeroom]{cancel}

\usepackage[margin=0.5in]{geometry}
\newcommand{\overbar}[1]{\mkern 1.5mu\overline{\mkern-1.5mu#1\mkern-1.5mu}\mkern 1.5mu}

\newcommand*\conj[1]{\overbar{#1}}
% STATS Shortcut
\newcommand{\BP}{\Bbb P}
\newcommand{\BR}{\Bbb R}
\let\Bbb\mathbb
\def\sep{\phantom{}}
\newcommand\todo[1]{\phantom{#1}}
\theoremstyle{definition}
\newtheorem{definition}{Définition}[section]
\newtheorem*{example}{Exemple}
\newtheorem{theorem}{Theorême}[section]
\newtheorem{corollary}{Corollaire}[theorem]
\newtheorem{lemma}[theorem]{Lemme}  

\title{Théorie des groupes}

\begin{document}
\chapter{Introduction}
Qu'est-ce qu'un groupe?
\chapter{Les Opérations}
Il est tout d'abord nécessaire d'aborder le principe d'\textbf{opérations} avant de parler de groupes à proprement parlé.
On connaît généralement le principe de ce qu'est une opération plutôt intuitivement, mais on pourrait donner plusieurs exemples d'opérations:
l'\textbf{addition} (dénoté avec l'opérateur +), la soustraction (dénoté avec l'opérateur -) ou la multiplication (dénoté par $\cdot$).

Évidemment une opération n'est pas simplement défini pour l'arithmétie. L'opérateur $\subset$ dénote une opération de sous-ensemble sur les ensembles par exemple.

Il est aussi possible de créer de nouveaux opérateurs et de nouvelles opérations, mais tout d'abord, commençons par définir les propriétés et les
définitions entourant le principe d'opérations. Dans le cadre de cet ouvrage, nous commencerons par l'\textbf{opération binaire}, un type
d'opération qui nous permettra d'établir la définition de ce qu'est un groupe.
\section{L'Opération Binaire}

\begin{definition}[Opération binaire]
    \label{def:operation_binaire}
    Soit $S$, un ensemble non-vide, toute \hyperref[def:fonction]{fonction} dont le domaine est l'ensemble des \hyperref[def:relations]{relations} sur $S$ et l'image est $S$ tel que
    
    $$\ast : S \times S \mapsto S$$ est une \textbf{opération binaire}.
\end{definition}

On en comprend qu'une opération binaire:
\begin{itemize}
    \item Existe dans le cadre d'un ensemble non-vide.
    \item Que c'est une \hyperref[def:fonction]{fonction} qui prend essentielement deux valeurs et en retourne une.
    \item Que les composants de la \hyperref[def:pre_image]{préimage} et de l'\hyperref[def:image]{image} proviennent nécessairement du même ensemble.
    \item Que toutes les combinaisons possibles $X \times X$ sont définies pour cette fonction
\end{itemize}

\todo{Add Examples}

\section{Principe de fermeture}
\todo{Add lien avec topologie}
\begin{definition}[Fermeture de l'opération binaire]
    \label{def:fermeture}
    Soit $\ast$ une \hyperref[def:operation_binaire]{opération binaire} sur $S$ et $A$ un \hyperref[def:sous_ensemble]{sous-ensemble} de $S$. 
    $A$ est \textbf{fermé} sous $\ast$ si et seulement si $\forall a,b \in A, a \ast b \in A$
\end{definition}
En d'autres mots, $A$ représente un sous-ensemble de $S$ pour lesquels chacun de ces éléments dont on applique l'opération binaire
produira un élément aussi présent dans $A$.

Par la définiton même de l'\hyperref[def:operation_binaire]{opération binaire}, $S$ est considéré fermé sous $\ast$. Donc si $\ast$
est considéré comme une opération binaire sur $S$, elle est nécessairement fermée par définition.
\todo{Ajouter example table de Caley}

\section{Associativité}
Généralement, afin d'être utile en mathématiques, on cherche à travailler avec des opérateurs dit \textbf{associatifs}. Cette 
propriété nous permet d'abstraire l'opérateur de son fonctionnement. Tout d'abord la définition:
\begin{definition}[Associativité]
    \label{def:associativite}
    Soit $\ast$ une \hyperref[def:operation_binaire]{opération binaire} sur $S$, $\ast$ est associative si et seulement si 
    $$\forall a,b,c \in S, a \ast (b \ast c) = (a \ast b) \ast c$$
\end{definition}
Ainsi, peu importe l'ordre dans laquelle on applique l'opération donnera le même résultat. Par exemple, l'addition est associative alors
que la soustraction ne l'est pas.
\section{Commutativité}
La commutativité, souvent de paire avec l'associativité est une propriété pratique lorsque l'on fait de l'algèbre, mais en aucun cas requise.
\begin{definition}[Commutativité]
    \label{def:commutativite}
    Soit $\ast$ une \hyperref[def:operation_binaire]{opération binaire} sur $S$, $\ast$ est commutative si et seulement si 
    $$\forall a,b \in S, a \ast b = b \ast a$$
\end{definition}
Ainsi, peu importe l'ordre des paramètres passées à la fonction ou à l'opération, le résultat sera le même.
L'addition est commutative, mais la composition de fonction ne l'est pas.

\section{Élément neutre}
La présence d'un élément neutre dans une opération indique qu'il existe un élément pour lequel s'il est utilisé comme 
paramètre de la fonction, redonnera la même valeur que l'autre paramètre. 

\begin{definition}[Élément neutre]
    \label{def:neutre}
    Soit $e$ un élément de $S$ et $\ast$ une opération binaire de $S$. Si $$\forall a \in S, a \ast e = e \ast a = a$$
    alors $e$ est un \textbf{élément neutre}.
\end{definition}
Un exemple classique est 1 pour la multiplication et le 0 pour l'addition.

\begin{theorem}
    \label{thm:neutre_unique} Soit $e$ un \hyperref[def:neutre]{élément neutre} de $S$ sur $\ast$, alors $e$ est unique.
\end{theorem}

\begin{proof}
    Soit $e$ et $e'$, deux \hyperref[def:neutre]{éléments neutres} de $S$ sur $\ast$. \sep
    $\forall a \in S, a \ast e = a$ \sep et $\forall a \in S, e' \ast a = a$. \sep
    Il en suit que $e' \ast e = e'$ et $e' \ast e = e$ \sep.
    Donc $e' = e' \ast e = e$, $e' = e$.
\end{proof}
\begin{comment}
    On suppose deux éléments neutres différents afin de prouver qu'il n'existe
    qu'une seule valeur possible et qu'en fait, en bout de ligne $e = e'$. \sep 
    On utilise ici la définiton même de ce qu'est un élément neutre pour $e$. \sep 
    Maintenant, on le fait pour $e'$ \sep 
    On utilise le cas particulier de $a = e$ (puisque c'est $\forall a \in S$) et $a = e'$ \sep 
    On en conclut que $e = e'$ et est donc unique.
\end{comment}
\section{Élément inverse}

\chapter{Structures algébriques}

\section{Semi-groupe}
\section{Monoïde}
\section{Groupe}
\chapter{Propriétés des groupes}
\appendix
\chapter{Notions préalables}
Voici une série de définition importantes à considérer lors de la lecture de cet ouvrage. Celles-ci servent
de références pour les exemples et les preuves qui seront amenés dans les chapitres suivants. Pour une explication
plus complète et rigoureuse, il sera nécessaire de se réferrer aux livres précédents

\begin{definition}[Sous-ensemble]
    \label{def:sous_ensemble}
\end{definition}

\begin{definition}[Pré-image]
    \label{def:pre_image}
\end{definition}

\begin{definition}[Image]
    \label{def:image}
\end{definition}

\begin{definition}[Relation]
    \label{def:relation}
\end{definition}

\begin{definition}[Produit cartésien]
    \label{def:produit_cartesien}
\end{definition}

\begin{definition}[Fonction]
    \label{def:fonction}
\end{definition}


\end{document}