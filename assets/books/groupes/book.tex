\documentclass[12pt]{book}

\usepackage[utf8]{inputenc} % Required for inputting international characters
\usepackage[T1]{fontenc} % Output font encoding for international characters

\usepackage{comment}
\usepackage{mathpazo} % Palatino font
\usepackage{amssymb}
\usepackage{amsmath}
\usepackage{array}
\usepackage{amsthm}
\usepackage{graphicx}
\usepackage{multirow}
\usepackage{relsize}
\usepackage{hyperref}
\usepackage{xcolor}
\hypersetup{
    colorlinks,
    linkcolor={red!50!black},
    citecolor={blue!50!black},
    urlcolor={blue!80!black}
}
\usepackage[makeroom]{cancel}

\usepackage[margin=0.5in]{geometry}
\newcommand{\overbar}[1]{\mkern 1.5mu\overline{\mkern-1.5mu#1\mkern-1.5mu}\mkern 1.5mu}

\newcommand*\conj[1]{\overbar{#1}}
% STATS Shortcut
\newcommand{\BP}{\Bbb P}
\newcommand{\BR}{\Bbb R}
\let\Bbb\mathbb
\def\sep{\phantom{}}
\newcommand\todo[1]{\phantom{#1}}
\theoremstyle{definition}
\newtheorem{definition}{Définition}[section]
\newtheorem*{example}{Exemple}
\newtheorem{theorem}{Theorême}[section]
\newtheorem{corollary}{Corollaire}[theorem]
\newtheorem{lemma}[theorem]{Lemme}  

\title{Théorie des groupes}

\begin{document}
\chapter{Introduction}
Qu'est-ce qu'un groupe?
\chapter{Les Opérations}
Il est tout d'abord nécessaire d'aborder le principe d'\textbf{opérations} avant de parler de groupes à proprement parlé.
On connaît généralement le principe de ce qu'est une opération plutôt intuitivement, mais on pourrait donner plusieurs exemples d'opérations:
l'\textbf{addition} (dénoté avec l'opérateur +), la soustraction (dénoté avec l'opérateur -) ou la multiplication (dénoté par $\cdot$).

Évidemment une opération n'est pas simplement défini pour l'arithmétie. L'opérateur $\subset$ dénote une opération de sous-ensemble sur les ensembles par exemple.

Il est aussi possible de créer de nouveaux opérateurs et de nouvelles opérations, mais tout d'abord, commençons par définir les propriétés et les
définitions entourant le principe d'opérations. Dans le cadre de cet ouvrage, nous commencerons par l'\textbf{opération binaire}, un type
d'opération qui nous permettra d'établir la définition de ce qu'est un groupe.
\section{L'Opération Binaire}

\begin{definition}[Opération binaire]
    \label{def:operation_binaire}
    Soit $S$, un ensemble non-vide, toute \hyperref[def:fonction]{fonction} dont le domaine est l'ensemble des \hyperref[def:relations]{relations} sur $S$ et l'image est $S$ tel que
    
    $$\ast : S \times S \mapsto S$$ est une \textbf{opération binaire}.
\end{definition}

On en comprend qu'une opération binaire:
\begin{itemize}
    \item Existe dans le cadre d'un ensemble non-vide.
    \item Que c'est une \hyperref[def:fonction]{fonction} qui prend essentielement deux valeurs et en retourne une.
    \item Que les composants de la \hyperref[def:pre_image]{préimage} et de l'\hyperref[def:image]{image} proviennent nécessairement du même ensemble.
    \item Que toutes les combinaisons possibles $X \times X$ sont définies pour cette fonction
\end{itemize}

\todo{Add Examples}

\section{Principe de fermeture}
\todo{Add lien avec topologie}
\begin{definition}[Fermeture de l'opération binaire]
    \label{def:fermeture}
    Soit $\ast$ une \hyperref[def:operation_binaire]{opération binaire} sur $S$ et $A$ un \hyperref[def:sous_ensemble]{sous-ensemble} de $S$. 
    $A$ est \textbf{fermé} sous $\ast$ si et seulement si $\forall a,b \in A, a \ast b \in A$
\end{definition}
En d'autres mots, $A$ représente un sous-ensemble de $S$ pour lesquels chacun de ces éléments dont on applique l'opération binaire
produira un élément aussi présent dans $A$.

Par la définiton même de l'\hyperref[def:operation_binaire]{opération binaire}, $S$ est considéré fermé sous $\ast$. Donc si $\ast$
est considéré comme une opération binaire sur $S$, elle est nécessairement fermée par définition.
\todo{Ajouter example table de Caley}

\section{Associativité}
Généralement, afin d'être utile en mathématiques, on cherche à travailler avec des opérateurs dit \textbf{associatifs}. Cette 
propriété nous permet d'abstraire l'opérateur de son fonctionnement. Tout d'abord la définition:
\begin{definition}[Associativité]
    \label{def:associativite}
    Soit $\ast$ une \hyperref[def:operation_binaire]{opération binaire} sur $S$, $\ast$ est associative si et seulement si 
    $$\forall a,b,c \in S, a \ast (b \ast c) = (a \ast b) \ast c$$
\end{definition}
Ainsi, peu importe l'ordre dans laquelle on applique l'opération donnera le même résultat. Par exemple, l'addition est associative alors
que la soustraction ne l'est pas.
\section{Commutativité}
La commutativité, souvent de paire avec l'associativité est une propriété pratique lorsque l'on fait de l'algèbre, mais en aucun cas requise.
\begin{definition}[Commutativité]
    \label{def:commutativite}
    Soit $\ast$ une \hyperref[def:operation_binaire]{opération binaire} sur $S$, $\ast$ est commutative si et seulement si 
    $$\forall a,b \in S, a \ast b = b \ast a$$
\end{definition}
Ainsi, peu importe l'ordre des paramètres passées à la fonction ou à l'opération, le résultat sera le même.
L'addition est commutative, mais la composition de fonction ne l'est pas.
\todo{Ajouter définition commute}

\section{Élément neutre}
La présence d'un élément neutre dans une opération indique qu'il existe un élément pour lequel s'il est utilisé comme 
paramètre de la fonction, redonnera la même valeur que l'autre paramètre. 

\begin{definition}[Élément neutre]
    \label{def:neutre}
    Soit $e$ un élément de $S$ et $\ast$ une opération binaire de $S$. Si $$\forall a \in S, a \ast e = e \ast a = a$$
    alors $e$ est un \textbf{élément neutre}.
\end{definition}
Un exemple classique est 1 pour la multiplication et le 0 pour l'addition.

\subsection{Unicité de la neutralité}
Une propriété importante est l'unicité de l'élément neutre dans un monoïde.
\begin{theorem}
    \label{thm:neutre_unique} Soit $e$ un \hyperref[def:neutre]{élément neutre} de $S$ sur $\ast$, alors $e$ est unique.
\end{theorem}

\begin{proof}
    Soit $e$ et $e'$, deux \hyperref[def:neutre]{éléments neutres} de $S$ sur $\ast$. \sep
    $\forall a \in S, a \ast e = a$ \sep et $\forall a \in S, e' \ast a = a$. \sep
    Il en suit que $e' \ast e = e'$ et $e' \ast e = e$ \sep.
    Donc $e' = e' \ast e = e$, $e' = e$.
\end{proof}
\begin{comment}
    On suppose deux éléments neutres différents afin de prouver qu'il n'existe
    qu'une seule valeur possible et qu'en fait, en bout de ligne $e = e'$. \sep 
    On utilise ici la définiton même de ce qu'est un élément neutre pour $e$. \sep 
    Maintenant, on le fait pour $e'$ \sep 
    On utilise le cas particulier de $a = e$ (puisque c'est $\forall a \in S$) et $a = e'$ \sep 
    On en conclut que $e = e'$ et est donc unique.
\end{comment}

\section{Élément inverse}
Un élément inverse dans une opération est un élément différent pour chaque élément de l'ensemble duquel l'opération 
resultera en l'élément neutre.
\begin{definition}[Élément inverse]
    \label{def:inverse}
    Soit $S$ un ensemble et $\ast$ un opération binaire sur $S$ avec un \hyperref[def:neutre]{élément neutre} $e$.

    Un élément $a$ est dit posséder un \textbf{élément inverse} $a^{-1}$ si et seulement si 
    $$ a \ast a^{-1} = a^{-1} \ast a = e $$
\end{definition}

Un exemple simple d'inverse est pour l'addition sur les $\Bbb R$: si $a = 2 \implies a^{-1} = -2$. En effet $2 + (-2) = 0$ (0 est l'élément neutre de l'addition).

\subsection{Inverse de l'élément neutre}
\begin{theorem}
    \label{thm:inverse_neutre} Soit $e$ un \hyperref[def:neutre]{élément neutre} et $e^{-1}$, l'\hyperref[def:inverse]{inverse} de $e$ pour l'opération binaire $\ast$,
    alors $e = e^{-1}$.
\end{theorem}
\begin{proof}
    Par la définition de l'\hyperref[def:inverse]{inverse}: $ e^{-1} \ast e = e $. \\ \sep 
    Par la définition de l'\hyperref[def:neutre]{élément neutre}: $ e^{-1} \ast e = e^{-1} $ \\ \sep
    Donc: $e^{-1} = e$
\end{proof}

\subsection{Unicité de l'inversibilité}
\begin{theorem}
    \label{thm:inverse_unique} Soit $a$ et $a^{-1}$, son inverse sur $\ast$, une \hyperref[def:operation_binaire]{opération binaire} \hyperref[def:associativite]{associative}, $a^{-1}$ est unique.
\end{theorem}
\begin{proof}
    Soit $a$, $a_1^{-1}$, $a_2^{-1}$ et $e$, un élément quelconque, deux \hyperref[def:inverse]{éléments inverses} et l'\hyperref[def:neutre]{élément neutre} sur $\ast$. \sep \\
    Par définition de l'inverse: $a \ast a_1^{-1} = e$ et $a_2^{-1} \ast a = e$. \sep \\
    $a_2^{-1} \ast e = a_2^{-1}$ \sep \\
    $\implies a_2^{-1} \ast (a \ast a_1^{-1}) = a_2^{-1}$ \sep \\
    $\implies (a_2^{-1} \ast a) \ast a_1^{-1} = a_2^{-1}$ \sep \\
    $\implies e \ast a_1^{-1} = a_2^{-1}$ \sep \\
    $\implies a_1^{-1} = a_2^{-1}$.
    Donc l'inverse est unique.
\end{proof}

Attention, cela ne veut pas dire que $a^{-1}$ est distinct! Il est possible que $a = a^{-1}$ par exemple, voir l'exemple \hyperref[thm:inverse_neutre]{ci-haut}.



\chapter{Structures algébriques}
Commençons par définir ce qu'on entend par structure algébrique:
\begin{definition}[Structure algébrique]
    \label{def:structure_algebrique}
    Une \textbf{structure algébrique} $(S, \ast, \dots)$ est un ensemble \textit{non vide} d'éléments avec une ou plusieurs
    operations binaires définies sur cet ensemble.
\end{definition}
Notons que pour toute structure algébrique:
\begin{itemize}
    \item L'ensemble $S$ doit être non vide
    \item Être définie pour l'ensemble complet par ses opérateurs, surjective sur ses opérateurs \todo{Add link}.
    \item En d'autres mot, les opérateurs sont \hyperref[def:fermeture]{fermés} sur $S$
\end{itemize}

Il existe différents types de structure algébrique possédant différentes propriétés et définis par certaines
caractéristiques sur l'ensemble $S$ ou sur les operations binaires le composant.

\section{Semi-groupe}
\begin{definition}[Semi-groupe]
    \label{def:semi_groupe}
    Un \textbf{semi-groupe} est une \hyperref[def:structure_algebrique]{structure algébrique} dont l'\hyperref[def:operation_binaire]{opérateur} est 
    \hyperref[def:associativite]{associatif}.
\end{definition}
Afin de prouver qu'une structure algébrique soit un semi-groupe, il faut:
\begin{itemize}
    \item Prouver que l'opérateur $\ast$ est fermé sur $S$. (et donc que $S$ est aussi non-vide)
    \item Prouver que l'opérateur $\ast$ est associatif.
\end{itemize}

\section{Monoïde}
\begin{definition}[Monoïde]
    \label{def:monoide}
    Un \textbf{monoïde} est un \hyperref[def:semi_groupe]{semi-groupe} (une structure algébrique associative) avec un \hyperref[def:neutre]{élément neutre}.
\end{definition}
Afin de prouver qu'une structure algébrique soit un monoïde, il faut:
\begin{itemize}
    \item Prouver que l'opérateur $\ast$ est fermé sur $S$. (et donc que $S$ est aussi non-vide)
    \item Prouver que l'opérateur $\ast$ est associatif.
    \item Trouver l'élément neutre (ou prouver qu'il existe).
\end{itemize}
\todo{Add exemple 2.26}
\begin{example}
    Prouvons que l'ensemble des paires ordonnées $S = \{(a, b) | a, b \in \Bbb R\}$ et l'opération $\ast$ tel que $(a, b) \ast (c, d) = (a + c, b + d + 2bd)$ est un monoïde. \sep \\
    1. L'opérateur $\ast$ est fermé, car le résultat de l'opération $\ast$ est $\in \Bbb R$ pour la paire ordonnée résultante.\sep 
    En effet $a + c \in \Bbb R$, car $a, c \in \Bbb R$ est l'addition de deux nombres réels est un réel. \sep 
    L'argument est similaire pour $b + d + 2bd$, car $b, d, 2 \in \Bbb R$. \sep \\
    2.L'opérateur est associatif, car $(a_1,b_1) \ast ( (a_2, b_2) \ast (a_3, b_3) ) =  $ \sep $(a_1,b_1) \ast (a_2 + a_3, b_2 + b_3 + 2b_2b_3) =$ \sep  $(a_1 + a_2 + a_3, b_1 + b_2 + b_3 + 2b_2b_3 + 2b_1b_2 + 2b_1b_3 + 4b_1b_2b_3)$ \sep 
    et $( (a_1,b_1) \ast (a_2, b_2)  ) \ast (a_3, b_3) = $ \sep $ (a_1 + a_2, b_1 + b_2 + 2b_1b_2) \ast (a_3, b_3) = $ \sep $(a_1 + a_2 + a_3, b_3 + b_1 + b_2 + 2b_1b_2 + 2b_3b_1 + 2b_3b_2 + 4b_1b_2b_3)$ \\
    3.On retrouve l'élément neutre en $(0, 0)$, car $(a_1, b_1) \ast (0, 0) =$ \sep $(a_1 + 0, b_1 + 0 + 2b_10) = $ \sep $(a_1, b_1)$
\end{example}

\section{Groupe}
\begin{definition}[Groupe]
    \label{def:groupe}
    Un \textbf{groupe} est un \hyperref[def:monoide]{monoïde} (une structure algébrique associative avec un élément neutre), noté $(G, \ast)$, dont chaque élément possède un \hyperref[def:inverse]{inverse}.
\end{definition}
Afin de prouver qu'une structure algébrique soit un groupe, il faut:
\begin{itemize}
    \item Prouver que l'opérateur $\ast$ est fermé sur $S$. (et donc que $S$ est aussi non-vide)
    \item Prouver que l'opérateur $\ast$ est associatif.
    \item Trouver l'élément neutre (ou prouver qu'il existe).
    \item Trouver ou prouver que chaque élément $a \in G$ possède un inverse $a^{-1}$
\end{itemize}
\begin{example}
    Soit $a \in \Bbb R - \{0, 1\}$ et $G_a = \{a^k | k \in \Bbb Z\}$. Prouvons que c'est un groupe avec $\cdot$, la multiplication habituelle: \sep \\
    1. La fermeture: $a^{k1} \cdot a^{k2} = a^{k1 + k2}$. Or, $(k_1 + k_2) \in \Bbb Z \implies a^{k1 + k2} \in G_a$ \sep \\
    2. L'associativité: $a^{k1} \cdot (a^{k2} \cdot a^{k3}) =$ \sep 
    $a^{k1} \cdot (a^{k2 + k3}) = a^{k1 + k2 + k3}$ et \sep 
    $(a^{k1} \cdot a^{k2}) \cdot a^{k3} = a^{k1 + k2} \cdot a^{k3} = a^{k1 + k2 + k3}$ \sep \\
    3. L'élément neutre: Posons $e = a^0$, en effet $a^k \cdot a^0 = a^0 \cdot a^k = a^k \cdot 1 = a^k$ \sep \\
    4. L'inverse: Posons $a^{-k}$, ainsi, $a^k \cdot a^{-k} = a^{k + (-k)} = a^0 = e$ 
\end{example}

\section{Groupe abélien}
\begin{definition}[Groupe abélien]
    \label{def:groupe_abelien}
    Un \textbf{groupe abélien} ou \textbf{groupe commutatif} est un \hyperref[def:groupe]{groupe} 
    (une structure algébrique associative avec un élément neutre et un élément inverse $\exists a^{-1} \forall a \in G$), 
    dont l'opération binaire est \hyperref[def:commutativite]{commutative}.
\end{definition}
Afin de prouver qu'une structure algébrique soit un groupe, il faut:
\begin{itemize}
    \item Prouver que l'opérateur $\ast$ est fermé sur $S$. (et donc que $S$ est aussi non-vide)
    \item Prouver que l'opérateur $\ast$ est associatif.
    \item Prouver que l'opérateur $\ast$ est commutatif.
    \item Trouver l'élément neutre (ou prouver qu'il existe).
    \item Trouver ou prouver que chaque élément $a \in G$ possède un inverse $a^{-1}$
\end{itemize}


\chapter{Propriété des groupes}
Maintenant que les définitions axiomatiques ont été données, nous pouvons commencer á donner les propriétés intéressantes
des groupes.

\section{Inversibilité dans un groupe}
\begin{theorem}
    \label{thm:groupe_inversible}
    Soit $(G, \ast)$ un groupe et $a, b \in G$, alors $(a \ast b)^{-1} = b^{-1} \ast a^{-1}$
\end{theorem}
\begin{proof}
    Posons $c = (a \ast b)$ et $d = b^{-1} \ast a^{-1}$. \sep
    $c, d \in G$ par le principe de fermeture. \sep \\
    $c^{-1} \ast c = e$ et $d \ast d^{-1} = e$ \sep \\
    $\implies (c^{-1} \ast (a \ast b)) \ast ((b^{-1} \ast a^{-1}) \ast d^{-1}) = e$ \sep \\
    $\implies (c^{-1} \ast a) \ast (b \ast b^{-1}) \ast (a^{-1} \ast d^{-1}) = e$ \sep \\
    $\implies (c^{-1} \ast a) \ast e \ast (a^{-1} \ast d^{-1}) = e$ \sep \\
    $\implies c^{-1} \ast (a \ast a^{-1}) \ast d^{-1} = e$ \sep \\
    $\implies c^{-1} \ast e \ast d^{-1} = e$ \sep \\
    $\implies c^{-1} \ast d^{-1} = e$ \sep \\
    $d^{-1}$ est l'inverse de $c^{-1}$ et puisque l'élément inverse est unique et $c$ est l'inverse de $c^{-1}$, $c = d^{-1}\implies (a \ast b)^{-1} = b^{-1} \ast a^{-1}$
\end{proof}
\section{Loi d'associativité généralisée}
\section{Loi de simplification}
\section{Unicité des solutions}
\section{Cardinalité d'un groupe}
\section{Loi des exposants}

\appendix
\chapter{Notions préalables}
Voici une série de définition importantes à considérer lors de la lecture de cet ouvrage. Celles-ci servent
de références pour les exemples et les preuves qui seront amenés dans les chapitres suivants. Pour une explication
plus complète et rigoureuse, il sera nécessaire de se réferrer aux livres précédents

\begin{definition}[Sous-ensemble]
    \label{def:sous_ensemble}
\end{definition}

\begin{definition}[Pré-image]
    \label{def:pre_image}
\end{definition}

\begin{definition}[Image]
    \label{def:image}
\end{definition}

\begin{definition}[Relation]
    \label{def:relation}
\end{definition}

\begin{definition}[Produit cartésien]
    \label{def:produit_cartesien}
\end{definition}

\begin{definition}[Fonction]
    \label{def:fonction}
\end{definition}

\chapter{Cheatsheet}
Voici ici un résumé de certaines propriétés importantes d'objets mathématique. Utile lorsque l'on désire une vue d'ensemble.
\section{Groupe}
Un \hyperref[def:groupe]{groupe} est un type de \hyperref[def:monoide]{monoide} qui est un type de \hyperref[def:semi_groupe]{semi-groupe} qui est une \hyperref[def:structure_algebrique]{structure algébrique}
, s'écrit $(G, \ast)$ et possède les propriétés suivantes:
\begin{itemize}
    \item $\ast$ est une \hyperref[def:operation_binaire]{opération binaire} \hyperref[def:associativite]{associative} et \hyperref[def:fermeture]{fermée} sur $G$.
    \item Si $\ast$ est \hyperref[def:commutativite]{commutative}, alors on dit que le groupe est un \hyperref[def:groupe_abelien]{groupe abélien}.
    \item Possède un \hyperref[def:neutre]{élément neutre} \hyperref[thm:neutre_unique]{unique}.
    \item Possède pour chaque élément de $G$ un \hyperref[def:inverse]{inverse} \hyperref[thm:inverse_unique]{unique}.
    \item L'inverse du résultat de l'opération = l'opération sur les inverses. Théorême.
    \item Possède la possibilité de se simplifier: $a \ast c = b \ast c \implies a = b$.
    \item Les solutions aux équations sont uniques.
    \item Possède au moins six éléments si non-commutatif.
    \item Obéissent aux lois des exposants.
\end{itemize}
\todo{Add links to def and thm}

\end{document}