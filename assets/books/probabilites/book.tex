\documentclass[12pt]{book}

\usepackage[utf8]{inputenc} % Required for inputting international characters
\usepackage[T1]{fontenc} % Output font encoding for international characters

\usepackage{comment}
\usepackage{mathpazo} % Palatino font
\usepackage{amssymb}
\usepackage{amsmath}
\usepackage{array}
\usepackage{amsthm}
\usepackage{graphicx}
\usepackage{multirow}
\usepackage{relsize}
\usepackage{hyperref}
\usepackage{xcolor}
\hypersetup{
    colorlinks,
    linkcolor={red!50!black},
    citecolor={blue!50!black},
    urlcolor={blue!80!black}
}
\usepackage[makeroom]{cancel}

\usepackage[margin=0.5in]{geometry}
\newcommand{\overbar}[1]{\mkern 1.5mu\overline{\mkern-1.5mu#1\mkern-1.5mu}\mkern 1.5mu}

\newcommand*\conj[1]{\overbar{#1}}
% STATS Shortcut
\newcommand{\BP}{\Bbb P}
\newcommand{\BR}{\Bbb R}
\let\Bbb\mathbb
\def\sep{\phantom{}}
\newcommand\todo[1]{\phantom{#1}}
\theoremstyle{definition}
\newtheorem{definition}{Définition}[section]
\newtheorem*{example}{Exemple}
\newtheorem{theorem}{Theorême}[section]
\newtheorem{corollary}{Corollaire}[theorem]
\newtheorem{lemma}[theorem]{Lemme}  

\title{Probabilités}

\begin{document}

\chapter{Variable aléatoire}
\begin{definition}[Variable aléatoire]
    \label{def:variable_aleatoire}
    Soit $X$, une \textbf{variable aléatoire} et $\Omega$, l'ensemble de toutes les valeurs possibles
    d'une expérience aléatoire.

    $X$ est un fonction tel que:
    $$ X: \Omega \to \Bbb R$$
\end{definition}

\begin{example}
    Prenons l'expérience qui consiste à lancer deux dés:
    $$ \Omega = {(\omega_1, \omega_2): \omega_1 = 1 \dots 6, \omega_2 = 1 \dots 6} $$
    La variable aléatoire $X$ définie comme la somme des deux dés:
    $$ X : \Omega \to {2, \dots, 12} \subset \Bbb R; (\omega_1, \omega_2) \mapsto \omega_1 + \omega_2 $$
\end{example}
\end{document}

 f(x) = \frac{x^5}{20}-\frac{3 x^4}{4}+\frac{17 x^3}{4}-\frac{45 x^2}{4}+\frac{117 x}{10}+6