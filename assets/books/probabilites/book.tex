\documentclass[12pt]{book}

\usepackage[utf8]{inputenc} % Required for inputting international characters
\usepackage[T1]{fontenc} % Output font encoding for international characters

\usepackage{comment}
\usepackage{mathpazo} % Palatino font
\usepackage{amssymb}
\usepackage{amsmath}
\usepackage{array}
\usepackage{amsthm}
\usepackage{graphicx}
\usepackage{multirow}
\usepackage{relsize}
\usepackage{hyperref}
\usepackage{xcolor}
\hypersetup{
    colorlinks,
    linkcolor={red!50!black},
    citecolor={blue!50!black},
    urlcolor={blue!80!black}
}
\usepackage[makeroom]{cancel}

\usepackage[margin=0.5in]{geometry}
\newcommand{\overbar}[1]{\mkern 1.5mu\overline{\mkern-1.5mu#1\mkern-1.5mu}\mkern 1.5mu}

\newcommand*\conj[1]{\overbar{#1}}
% STATS Shortcut
\newcommand{\BP}{\Bbb P}
\newcommand{\BR}{\Bbb R}
\let\Bbb\mathbb
\def\sep{\phantom{}}
\newcommand\todo[1]{\phantom{#1}}
\theoremstyle{definition}
\newtheorem{definition}{Définition}[section]
\newtheorem*{example}{Exemple}
\newtheorem{theorem}{Theorême}[section]
\newtheorem{corollary}{Corollaire}[theorem]
\newtheorem{lemma}[theorem]{Lemme}  

\title{Théorie des groupes}

\begin{document}
\chapter{Introduction}
Qu'est-ce que les probablités?

\chapter{Terminologie}
On tente ici de définir plusieurs termes utilisés en statistique. En effet, il est nécessaire d'introduire
certaines définitions afin de pouvoir bien établir ce dont on parle.

\section{Expérience probabiliste}
\begin{definition}[Expérience aléatoire]
    \label{def:exp_ale}
    Soit $e$ un élément de $S$ et $\ast$ une opération binaire de $S$. Si $$\forall a \in S, a \ast e = e \ast a = a$$
    alors $e$ est un \textbf{élément neutre}.
\end{definition}



\end{document}